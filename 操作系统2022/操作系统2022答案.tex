\documentclass[UTF8]{ctexart}
\usepackage{subfigure}
\usepackage{caption}
\usepackage{amsmath,bm}
\usepackage{amssymb}
\usepackage{pifont}
\usepackage{geometry}
\usepackage{graphicx}
\usepackage{gensymb}
\usepackage{wrapfig}
\usepackage{titlesec}
\usepackage{float}
\usepackage{diagbox}
\usepackage{fancyhdr}
\usepackage{color}
\usepackage{bm}
\usepackage{siunitx}
\usepackage{ulem}
\usepackage{CJKulem}
\pagestyle{plain}
\geometry{a4paper,scale=0.8}
\CTEXsetup[format+={\raggedright}]{section} 
\title{操作系统2022答案}
\author{Deschain}
\titlespacing*{\section}
{0pt}{0pt}{0pt}
\titlespacing*{\subsection}
{0pt}{0pt}{0pt}
\titlespacing*{\paragraph}
{0pt}{0pt}{0pt}
\titlespacing*{\subparagraph}
{0pt}{0pt}{0pt}
\titleformat*{\section}{\normalsize}
\titleformat*{\subsection}{\normalsize}
\begin{document}
\maketitle
\section*{一、}
1.C 2.B 3.B 4.D 5.A 6.B 7.D 8.D 9.C 10.A \\
\section*{二、}
\subsection*{1.}
运行态0~4个,就绪态0~6个,阻塞态0~10个。\\
\subsection*{2.}
$10ms+60s\div 5400\div 2=15.6ms$\\
\subsection*{3.}
共有$20GB\div 4KB=5M$个簇,因此位图大小为5M,需要$5M\div 4KB=5M\div 32K=160$个簇来存放。\\
\subsection*{4.}
单缓冲区:$(100+10)\times 100=11000\mu s=11ms$\\
双缓冲区:$(100+10)\times 100=11000\mu s=11ms$\\
\subsection*{5.}
$\frac{20}{100}+\frac{40}{150}+\frac{100}{300}=0.8<1$,可调度\\
\subsection*{6.}
$16\times 0.5ms=8ms$,中断处理方式是合理的,因为鼠标需要的时间占比很小。\\
\subsection*{7.}
$\lfloor \frac{12345}{4\times 1024} \rfloor = 3 $,页内偏移量57,物理地址$6\times 4\times 1024+57=24633$\\
\subsection*{8.}
$\lfloor \frac{46000}{2\times 1024} \rfloor = 22 $,块内偏移量944,位于第$92+22=114$块。\\
\section*{三、}
\subsection*{1.}
\begin{tabular}{|c|c|c|c|c |c|c|c|c|c |c|c|c|c|c |c|c|c|c|c |c|}
\hline
FCFS &A&A&A&A&A&A&B&B&B&C&C&C&C&D&D&D&D&D&E&E\\
\hline
SJF  &A&B&B&B&C&E&E&C&C&C&D&D&D&D&D&A&A&A&A&A\\
\hline
HRRN &A&A&A&A&A&A&B&B&B&E&E&C&C&C&C&D&D&D&D&D\\
\hline
SRT  &A&B&B&B&C&E&E&C&C&C&A&A&A&A&A&D&D&D&D&D\\
\hline
\end{tabular}
\newline
周转时间:\\
\begin{tabular}{|c|c|c|c|c |c|c|}
\hline
$\quad$&A &B &C &D &E &系统 \\
\hline
FCFS   &6 &8 &11&14&15&10.8 \\ 
\hline
SJF    &20&3 &8 &11&2 &8.8  \\
\hline
HRRN   &6 &8 &13&16&6 &9.8  \\
\hline
SRT    &15&3 &8 &16&2 &8.8  \\
\hline
\end{tabular}
\newline
\subsection*{2.}
\begin{tabular}{|c|c|c|c|c |c|c|c|c|c |c|c|c|c| }
    \hline
    OPT  &M&M&M&H&M&H&M&H&H&M&H&H&6次\\
    \hline
    FIFO &M&M&M&H&M&H&M&M&M&M&H&M&9次\\
    \hline
    LRU  &M&M&M&H&M&M&M&M&H&M&H&M&9次\\
    \hline
    CLOCK&M&M&M&H&M&M&M&M&H&M&H&M&9次\\
    \hline 

\end{tabular}
\subsection*{3.}
(a)$256^2\times 2\times 100ns=13107200ns=13.1072ms$\\
(b)共用到$256^2\times 4/(4\times 1024)=64$页,初次访问缺页,因此TLB的命中率为$1-64/(256^2)=99.902\%$\\
(c)$64-16=48$次\\
(d)$256^2\times(10ns+100ns)+64\times 100ns=7215360ns=7.21536ms$\\
\subsection*{4.}
typedef Semaphore int;\\
Semaphore Guard = 1;//保安空闲\\
Semaphore GuardQueue = 0;//等待保安扫码的人数\\
Semaphore Inside = 5;//储蓄所可进人数\\
Semaphore Counter = 1;//柜员空闲\\
Semaphore CounterQueue = 0;//等待柜员服务的人数\\
Semaphore door = 1;//闸机空闲\\
void CustomerFunc()\{\\
V(GuardQueue); //通知保安,有人等待扫码\\
P(Guard);//申请保安服务\\
扫码;\\
P(Inside);//在入口处申请进门\\
进入储蓄所;\\
V(CounterQueue);//通知柜员,有人等待服务\\
P(Counter);//申请柜员服务\\
接受柜员服务;\\
P(Door);//申请使用闸机\\
出门;\\
V(Door);\\
V(Inside);\\
\}\\
void GuardFunc()\{\\
P(GuardQueue);\\
扫码;\\
V(Guard);  \\  
\}\\
void CounterFunc()\{\\
P(CounterQueue);\\
服务;\\
V(Counter);\\
\}\\











\end{document}