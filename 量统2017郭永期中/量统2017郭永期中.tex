\documentclass[UTF8]{ctexart}
\usepackage{subfigure}
\usepackage{caption}
\usepackage{amsmath}
\usepackage{amssymb}
\usepackage{geometry}
\usepackage{graphicx}
\usepackage{gensymb}
\usepackage{wrapfig}
\usepackage{titlesec}
\usepackage{float}
\usepackage{diagbox}
\usepackage{fancyhdr}
\pagestyle{plain}
\geometry{a4paper,scale=0.8}
\CTEXsetup[format+={\raggedright}]{section} 
\title{量统2017郭永期中}
\author{Deschain}
\titlespacing*{\section}
{0pt}{0pt}{0pt}
\titlespacing*{\subsection}
{0pt}{0pt}{0pt}
\titlespacing*{\paragraph}
{0pt}{0pt}{0pt}
\titlespacing*{\subparagraph}
{0pt}{0pt}{0pt}
\titleformat*{\section}{\normalsize}
\begin{document}
\maketitle
\section*{一、(本题共36分,每小题6分)简答题}
1.解释概念:(1)德布罗意假设;(2)束缚态\\
(1)实物粒子具有波粒二象性。\\
(2)粒子在无穷远处出现的概率为0的状态。\\
2.设每个粒子可占据单粒子态$\phi_1,\phi_2,\phi_3$中的一个态,对于两个全同Bose子体系,直接给出体系
可能态的数目及相应的态函数。\\
共6种。
\begin{equation*}
  \begin{aligned}
     & \psi_1=\phi_1(1)\phi_1(2),\quad\psi_2=\phi_2(1)\phi_2(2),\quad\psi_3=\phi_3(1)\phi_3(2) \\
     & \psi_4=\frac{1}{\sqrt2}[\phi_1(1)\phi_2(2)+\phi_2(1)\phi_1(2)]                          \\
     & \psi_5=\frac{1}{\sqrt2}[\phi_1(1)\phi_3(2)+\phi_3(1)\phi_1(2)]                          \\
     & \psi_6=\frac{1}{\sqrt2}[\phi_2(1)\phi_3(2)+\phi_3(1)\phi_2(2)]                          \\
  \end{aligned}
\end{equation*}
3.已知厄米算符$\hat A,\hat B$,给出算符$(\hat A+i\hat B)^2$厄米性的条件。\\
\begin{equation*}
  \begin{aligned}
     & (\hat A+i\hat B)^2=\hat A^2+2i\hat A\hat B-\hat B^2                         \\
     & \therefore 2i\hat A\hat B=-2i\hat A\hat B,\quad \hat A\hat B+\hat B\hat A=0
  \end{aligned}
\end{equation*}
4.写出三维自由粒子及三维各向同性谐振子的各两组守恒量完全集。\\
(1)三维自由粒子:$[\hat H,\hat L^2,\hat L_z],[\hat H_x,\hat H_y,\hat H_z]$\\
(2)三维各向同性谐振子:$[\hat H,\hat p],[\hat H,\hat L^2,\hat L_z]$\\
5.设厄米算符$\hat A$和$\hat B$满足$\hat A^2=\hat B^2=1$($\hat A,\hat B$的本征值无简并),
直接给出:(1)$\hat A,\hat B$的本征值;(2)在$\hat A$表象中算符$\hat A$的矩阵表示。\\
(1)A,B的本征值都是$\lambda_1=1,\lambda_2=-1$\\
(2)
\begin{equation*}
  \begin{aligned}
    A=\begin{bmatrix}
      1 & 0  \\
      0 & -1
    \end{bmatrix}
  \end{aligned}
\end{equation*}
6.在$\hat L_z$的本征态$Y_{lm}(\theta,\varphi)$中,计算$\hat L_x$和$\hat L_y$的不确定度关系
$\Delta L_x\cdot\Delta L_y$。
\begin{equation*}
  [\hat L_x,\hat L_y]=i\hbar\hat L_z\quad
  \therefore\Delta\hat L_x\cdot\Delta\hat L_y\geq\frac{\hbar}{2}\overline{L_z}=\frac{m\hbar^2}{2}
\end{equation*}
\section*{二、(本题8分)(以下两题任选一题)}
1,在一维势场中运动的粒子,势能关于原点对称,即$U(-x)=U(x)$,证明粒子的定态波函数具有确定的宇称。\\
\begin{equation*}
  \begin{aligned}
     & E\psi(x)=-\frac{\hbar^2}{2m}\frac{d^2}{dx^2}\psi(x)+U(x)\psi(x)             \\
     & E\psi(-x)=-\frac{\hbar^2}{2m}\frac{d^2}{dx^2}\psi(-x)+U(-x)\psi(-x)         \\
     & \because U(-x)=U(x)\therefore\psi(x)=\psi(-x)\quad or\quad\psi(x)=-\psi(-x)
  \end{aligned}
\end{equation*}
2.证明定理:如果算符$\hat F$和$\hat G$有一组共同本征函数$\psi_n$,而且$\psi_n$组成完备系,则
$\hat F$和$\hat G$对易。
\begin{equation*}
  \begin{aligned}
     & \hat F\psi_n=F_n\psi_n,\quad\hat G\psi_n=G_n\psi_n                \\
     & (\hat F\hat G-\hat G\hat F)\psi_n=\hat FG_n\psi_n-\hat GF_n\psi_n
    =F_NG_n\psi_n-G_nF_n\psi_n=0                                         \\
     & \because\forall\psi,\psi=\sum C_n\psi_n                           \\
     & \therefore[\hat F,\hat G]\psi=0,\quad [\hat F,\hat G]=0
  \end{aligned}
\end{equation*}
\section*{三、(本题8分)设氢原子处于波函数$\psi(r,\theta,\varphi,s_z)=\frac{1}{\sqrt3}R_{21}(r)
  \begin{pmatrix}\sqrt2Y_{10}(\theta,\varphi)\\ Y_{11}(\theta,\varphi)\end{pmatrix}$}
描写的状态中,求力学量$\hat H,\vec L^2,\hat L_z,\hat S_z,\hat J_z$的可能取值,这些可能值出现的概率
和这些力学量的平均值。
\begin{equation*}
  \begin{aligned}
     & P(H=\frac{E_0}{4})=1,\quad\overline{H}=\frac{E_0}{4}                               \\
     & P(L^2=2\hbar^2)=1,\quad\overline{L^2}=2\hbar^2                                     \\
     & P(L_z=0)=\frac{2}{3},\quad P(L_z=\hbar)=\frac{1}{3},\quad
    \overline{L_z}=\frac{\hbar}{3}                                                        \\
     & P(S_z=\frac{\hbar}{2})=\frac{2}{3},\quad P(S_z=-\frac{\hbar}{2})=\frac{1}{3},\quad
    \overline S_z=\frac{\hbar}{6}                                                         \\
     & P(J_z=\frac{\hbar}{2})=1,\quad\overline J_z=\frac{\hbar}{2}                        \\
  \end{aligned}
\end{equation*}
\section*{四、(本题8分)一个量子“刚体”,具有惯性矩$I_z$,自由地在$xy$平面内转动,$\varphi$为转角。
  此系统的Hamilton量为$H=\frac{L_Z^2}{2I_z},L_z$为$z$方向的轨道角动量。在$t=0$时,转子由波包
  $\psi(0)=Asin^2\varphi$描述,求系统在$t>0$时的波函数$\psi(t)$。}
\begin{equation*}
  \begin{aligned}
     & \hat H=\frac{\hat L_z^2}{2I_z},\quad E_m=\frac{m^2\hbar^2}{2I_z},\quad
    \psi_{m1}=\frac{1}{\sqrt{2\pi}}e^{im\varphi},\quad
    \psi_{m2}=\frac{1}{\sqrt{2\pi}}e^{-im\varphi},\quad                       \\
     & \psi(0)=Asin^2\varphi=\frac{1}{\sqrt6}(\psi_{21}+\psi_{22}+\psi_0)     \\
     & \psi(t)=\frac{1}{\sqrt6}(\psi_{21}e^{\frac{im^2\hbar}{2I_z}t}
    +\psi_{22}e^{-\frac{im^2\hbar}{2I_z}t})+\sqrt\frac{2}{3}\psi_0
  \end{aligned}
\end{equation*}
\section*{五、(本题20分)}
1.(3分)直接写出在$\hat S_z$表象中电子自旋磁矩矩阵$\hat S_x,\hat S_y$和$\hat S_z$。\\
\begin{equation*}
  S_x=\frac{\hbar}{2}
  \begin{bmatrix}
    0 & 1 \\
    1 & 9
  \end{bmatrix}\quad
  S_y=\frac{\hbar}{2}
  \begin{bmatrix}
    0  & i \\
    -i & 0
  \end{bmatrix}\quad
  S_z=\frac{\hbar}{2}
  \begin{bmatrix}
    1 & 0  \\
    0 & -1
  \end{bmatrix}
\end{equation*}
2.(4分)求在$\hat S_z$表象中$\hat S_x$的本征值及相应的本征态。\\
$S_x$的本征值$\lambda_1=\frac{\hbar}{2},\lambda_2=-\frac{\hbar}{2}$,本征函数
$\psi_1=\frac{1}{\sqrt2}(1,1)^T,\psi_2=\frac{1}{\sqrt2}(1,-1)^T$\\
3.(4分)测量一个处于自由空间的电子自旋的$z$分量,结果为$\frac{\hbar}{2}$。问第二次测量自旋的$x$
分量,可能得到什么结果?得到这些结果的概率是多少?\\
\begin{equation*}
  P(S_x=\frac{\hbar}{2})=P(S_x=-\frac{\hbar}{2})=\frac{1}{2}
\end{equation*}
4.(4分)一个具有两个电子的原子,处于自旋单态$(S=0)$。讨论自旋-轨道耦合作用$\hat H=\xi(\vec r)
  \hat\vec S\cdot\hat\vec L$对能量的贡献。\\
\begin{equation*}
  \begin{aligned}
     & \hat{\vec J}=\hat{\vec S}+\hat{\vec L},\quad\hat{\vec S}\cdot\hat{\vec L}=\frac{1}{2}
    (\hat J^2-\hat S^2-\hat L^2)=\frac{1}{2}(\hat J^2-\hat L^2)                              \\
     & \hat H\lvert l>=\xi(r)\frac{1}{2}(\hat J^2-\hat L^2)\lvert l>
    =\xi(r)\frac{1}{2}(j^2-l^2)\lvert l>=0
  \end{aligned}
\end{equation*}
5.(5分)一系统由两个可区分的自旋$\frac{1}{2}$的粒子组成,实验测得粒子1的自旋投影总朝上(+z方向),
粒子2的自旋投影总是朝下(-z方向)。试求测量系统的总自旋平方$\vec S^2$及$S_z$得到的可能值和相应概率。
\begin{equation*}
  \lvert +->=\frac{1}{\sqrt2}(\lvert10>+\lvert00>),\quad
  P(\hat S^2=2\hbar^2)=\frac{1}{2},\quad P(\hat S^2=0)=\frac{1}{2},P(S_z=0)=1
\end{equation*}
\section*{六、(本题20分)已知二维谐振子的哈密顿算符为$\hat H_0=\frac{\hat{\vec p}^2}{2\mu}
    +\frac{1}{2}\mu\omega^2(x^2+y^2)$,对其施加微扰$\hat W=\lambda xy$($\lambda$为常数)。求解
  下列问题:}
1.(10分)求$\hat H_0$的本征值和本征态,并讨论各能级的简并度。\\
\begin{equation*}
  E_n=(n+\frac{1}{2})\hbar\omega,\quad\psi_n(x)=H_{n1}(\alpha x)H_{n2}(\alpha y)
  e^{-\frac{1}{2}\alpha^2(x^2+y^2)},\quad n=n_1+n_2,\quad f_n=n+1
\end{equation*}
2.(10分)利用定态微扰论求$\hat H=\hat H_0=\hat W$基态能量及第二激发态能量至一级修正。\\
参考公式:在$\hat H_0$表象中,坐标矩阵元为
\begin{equation*}
  x_{n'n}=<n'\lvert x\rvert n>=\sqrt\frac{\hbar}{2m\omega}[\sqrt{n+1}\delta_{n',n+1}
    +\sqrt n\delta_{n',n-1}]
\end{equation*}
$\lvert n>$为一维线性谐振子的第n个能量本征态,在坐标表象中即为$\lvert n>=\psi_n(x)$
\begin{equation*}
  \begin{aligned}
     & (1)E_0^{(0)}=\hbar\omega,\quad
    \psi_0(x)=H_0(\alpha x)H_0(\alpha y)e^{-\frac{1}{2}\alpha^2(x^2+y^2)}     \\
     & E_0^{(1)}=-\lambda<0\lvert x\rvert0><0\lvert y\rvert0>=0               \\
     & E_0=\hbar\omega                                                        \\
     & (2)E_2^{(0)}=3\hbar\omega,\quad\psi_1=\lvert20>,\quad\psi_2=\lvert11>,
    \quad\psi_3=\lvert02>                                                     \\
     & W_{11}=W_{22}=W_{33}=W_{13}=W_{31}=0,\quad
    W_{12}=W_{23}=W_{21}=W_{32}=-\frac{\lambda\hbar}{\sqrt2\mu\omega}         \\
     & \begin{vmatrix}
      E                                    & \frac{\lambda\hbar}{\sqrt2\mu\omega} & 0                                    \\
      \frac{\lambda\hbar}{\sqrt2\mu\omega} & E                                    & \frac{\lambda\hbar}{\sqrt2\mu\omega} \\
      0                                    & \frac{\lambda\hbar}{\sqrt2\mu\omega} & E
    \end{vmatrix}=0,\quad
    E(E-\frac{\lambda\hbar}{\mu\omega})(E+\frac{\lambda\hbar}{\mu\omega})=0   \\
     & E_{21}^{(1)}=0,\quad E_{22}^{(1)}=\frac{\lambda\hbar}{mu\omega},\quad
    E_{23}^{(1)}=-\frac{\lambda\hbar}{\mu\omega}
  \end{aligned}
\end{equation*}
\section*{七、附加题(本题10分)(以下三题任选其一)}
1.设体系在$t=0$时处于基态$\lvert 0>$,若$t=0$开始对体系施加微扰$\hat H'(x,t)=
  \hat F(x)e^{-\frac{t}{\tau}}$,长时间加上微扰后证明该体系处于另一能量本征态$\lvert 1>$的概率为
$\frac{\lvert<0\lvert\hat F\rvert 1>\rvert^2}{(E_1-E_0)^2+(\frac{\hbar}{\tau})^2}$。\\
2.谈谈你对《量子力学》的认识(200$\sim$300字以内)。\\
3.谈谈你对一种量子现象或效应的认识(200$\sim$300字以内)。




\end{document}