\documentclass[UTF8]{ctexart}
\usepackage{subfigure}
\usepackage{caption}
\usepackage{amsmath,bm}
\usepackage{amssymb}
\usepackage{pifont}
\usepackage{geometry}
\usepackage{graphicx}
\usepackage{gensymb}
\usepackage{wrapfig}
\usepackage{titlesec}
\usepackage{float}
\usepackage{diagbox}
\usepackage{fancyhdr}
\usepackage{color}
\usepackage{bm}
\usepackage{siunitx}
\usepackage{ulem}
\usepackage{CJKulem}
\pagestyle{plain}
\geometry{a4paper,scale=0.8}
\CTEXsetup[format+={\raggedright}]{section} 
\title{固物2013期中}
\author{Deschain}
\titlespacing*{\section}
{0pt}{0pt}{0pt}
\titlespacing*{\subsection}
{0pt}{0pt}{0pt}
\titlespacing*{\paragraph}
{0pt}{0pt}{0pt}
\titlespacing*{\subparagraph}
{0pt}{0pt}{0pt}
\titleformat*{\section}{\normalsize}
\begin{document}
\maketitle
\section*{\bfseries 一、填空}
1.GaAs的晶体结构是闪锌矿结构,其原胞中包含\uline{\mbox{\hspace{2em}}}个原子。\\
2.Si的晶体结构是\uline{\mbox{\hspace{4em}}},其布拉菲格子是\uline{\mbox{\hspace{6em}}}格子,与之对应的倒格子是
\uline{\mbox{\hspace{6em}}}格子。假设Si的晶格常数为$a$,则其布拉菲格子原胞的体积为\uline{\mbox{\hspace{3em}}},
而其倒格子的晶格常数为\uline{\mbox{\hspace{3em}}},第一布里渊区的体积是\uline{\mbox{\hspace{5em}}}。\\
3.对于全同原子组成的简单立方晶格、体心立方晶格和面心立方晶格而言,一个原子周围最近邻的原子个数分别是
\uline{\mbox{\hspace{2em}}},\uline{\mbox{\hspace{2em}}}和\uline{\mbox{\hspace{2em}}}。上述三种晶格中,堆积最
紧密的是\uline{\mbox{\hspace{6em}}}晶格,其原子排列最致密的等效晶面的密勒指数为\uline{\mbox{\hspace{5em}}}。\\
4.使用波长为$\lambda$的X射线对一晶体进行衍射,发现某一个一级衍射峰对应的衍射角为$2\theta$,则与之对应的一组平行晶面
的面间距可以表示为\uline{\mbox{\hspace{7em}}}。这组晶面对应倒格子空间的一个格点,因此该倒格点的格矢长度可以表示为
\uline{\mbox{\hspace{9em}}}。\\
5.晶体中的缺陷按照其几何特征可以分为\uline{\mbox{\hspace{2em}}}缺陷,\uline{\mbox{\hspace{2em}}}缺陷和
\uline{\mbox{\hspace{2em}}}缺陷。其中位错是一种典型的\uline{\mbox{\hspace{2em}}}缺陷。\\
6.原子的结合中,吸引作用主要来自于异性电荷之间的\uline{\mbox{\hspace{6em}}}作用,而排斥作用主要来自于同性电荷之间的
\uline{\mbox{\hspace{6em}}}作用以及\uline{\mbox{\hspace{7em}}}原理引起的排斥。当原子之间的距离大于平衡距离时,系
统的能量会随距离减小而\uline{\mbox{\hspace{3em}}};小于平衡距离时,系统的能量会随距离减小而
\uline{\mbox{\hspace{3em}}}。\\
7.金刚石材料中C-C键结合的方式是典型的\uline{\mbox{\hspace{6em}}},其电离度为\uline{\mbox{\hspace{2em}}}。\\
8.在索末菲自由电子模型中,自由电子在$k$空间的等能面为球面。假设金属晶体的总体积为$V$,则$k$空间$k$的状态密度为
\uline{\mbox{\hspace{5em}}},电子的状态密度为\uline{\mbox{\hspace{5em}}}。能量标度下自由电子的能态密度会随着能量
的增大而\uline{\mbox{\hspace{3em}}}。\\
9.布洛赫能带理论相比索末菲自由电子模型,主要是考虑了晶体中的\uline{\mbox{\hspace{7em}}}对电子运动的影响。在近自由电
子近似中,是以\uline{\mbox{\hspace{3em}}}电子的本征值和本征函数作为基态,将\uline{\mbox{\hspace{10em}}}看作微扰来
求解薛定谔方程的。\\
10.在能带底部,电子的平均速度\uline{\mbox{\hspace{3em}}},电子的有效质量\uline{\mbox{\hspace{3em}}};在能带顶部,
电子的平均速度\uline{\mbox{\hspace{3em}}},电子的有效质量\uline{\mbox{\hspace{3em}}}(均选填“=0”,“>0”或“<0”。)\\
\section*{\bfseries 二、简答}
1.倒格子空间的物理意义。\\
2.有效质量的物理意义。\\
3.晶体、非晶体和准晶体。\\
\section*{\bfseries 三、计算}
1.如图所示,是一个体心立方晶格的单胞,其晶格常数为$a$。\\
a.写出OAB晶面的密勒指数。\\
b.求OAB晶面的面间距。\\
2.相距为$r$的两原子,相互作用势能可以表示为$u(r)=u_0[(\frac{\sigma}{r})^12-\frac{\sigma}{r})^6]$,分别求出势能最小
和吸引力最强时的距离。\\
3.假设一个晶格常数为$a$的一维晶格中,电子能量$E(k)=E_0-2E_1cos(ka)$。
a.分别求能带底和能带顶处的简约波矢和能量。\\
b.分别求能带底和能带顶处电子的有效质量。\\
4.电子在一个晶格常数为$a$的一维晶体中运动。\\
a.求布里渊区边界$\frac{2\pi}{a}$处自由电子的能量。\\
b.假设单个电子感受到的周期性势场$V(x)=-V_0cos(\frac{4\pi x}{a})cos(\frac{2\pi x}{a})$,其中$V_0>0$,采用近自由电子
近似分别求出布里渊区边界$\frac{\pi}{a},\frac{2\pi}{a},\frac{3\pi}{a}$处的能隙。\\






\newpage
\section*{\bfseries 一、填空题答案}
1.\quad 2\\
2.\ding{172}金刚石\mbox{\hspace{2em}}
\ding{173}面心立方\mbox{\hspace{2em}}
\ding{174}体心立方\mbox{\hspace{2em}}
\ding{175}$\frac{a^3}{4}$\mbox{\hspace{2em}}
\ding{176}$\frac{4\pi}{a}$\mbox{\hspace{2em}}
\ding{177}$\frac{32\pi^3}{a^3}$\\
3.\ding{172}6\mbox{\hspace{2em}}
\ding{173}8\mbox{\hspace{2em}}
\ding{174}12\mbox{\hspace{2em}}
\ding{175}面心立方\mbox{\hspace{2em}}
\ding{176}$\{111\}$\\
4.\ding{172}$\frac{\lambda}{2sin(\theta)}$\mbox{\hspace{2em}}
\ding{173}$\frac{4\pi sin(\theta)}{\lambda}$\mbox{\hspace{2em}}
5.\ding{172}点\mbox{\hspace{2em}}
\ding{173}线\mbox{\hspace{2em}}
\ding{174}面\mbox{\hspace{2em}}
\ding{175}线\\
6.\ding{172}库仑(库仑吸引)\mbox{\hspace{2em}}
\ding{173}库仑(库仑排斥)\mbox{\hspace{2em}}
\ding{174}泡利不相容\mbox{\hspace{2em}}
\ding{175}减小\mbox{\hspace{2em}}
\ding{176}增大\\
7.\ding{172}共价键(共价结合)\mbox{\hspace{2em}}
\ding{173}0\\
8.\ding{172}$\frac{V}{8\pi^3}$\mbox{\hspace{2em}}
\ding{173}$\frac{V}{4\pi^3}$\mbox{\hspace{2em}}
\ding{174}增大\\
9.\ding{172}周期性势场\mbox{\hspace{2em}}
\ding{173}自由\mbox{\hspace{2em}}
\ding{174}周期性势场起伏(或周期性势场,或$\Delta V=V-V_0$)\\
10.\ding{172}$=0$\mbox{\hspace{2em}}
\ding{173}$>0$\mbox{\hspace{2em}}
\ding{174}$=0$\mbox{\hspace{2em}}
\ding{175}$<0$\\
\section*{\bfseries 二、简答题答案}
1.倒格子空间是正格子空间的傅里叶变换,倒格子空间是波矢空间(动量空间)。\\
正倒格子基矢之间的关系为:$\vec{a_i}\cdot\vec{b_j}=2\pi\delta_{ij}$。
2.有效质量是考虑了晶格周期性势场(晶格势场和其他电子的平均场)对电子作用后,电子在外场作用下表现出的等效质量。其关系式为:
$\frac{1}{m^*_{\alpha\beta}}=\frac{1}{\hbar^2}\frac{\partial^2E(k)}{\partial k_\alpha\partial k_\beta}$。\\
3.\ding{172}晶体:原子排列具有空间周期性。\\
\ding{173}非晶体:原子排列不具有空间周期性。\\
\ding{174}准晶体:原子排列不具有长程平移序,但具有长程取向序。\\
\section*{\bfseries 三、计算题答案}
1.a.$(11\overline{1})$\\
b.$\frac{\sqrt3}{6}a$\\
2.设势能最小的距离为$r_0$,吸引力最大的距离为$r_1$,则
\begin{equation*}
    \begin{aligned}
        &\frac{dU(r)}{dr}=6\sigma^6u_0(-\frac{2\sigma^6}{r^{13}}+\frac{1}{r^7})\\
        &\frac{dU(r)}{dr}\Bigg\lvert_{r=r_0}=0,\quad r_0=\sqrt[6]{2}\sigma\\
        &\frac{d^2U(r)}{dr^2}=6\sigma^6u_0(\frac{26\sigma^6}{r^{14}}-\frac{7}{r^8})\\
        &\frac{d^2U(r)}{dr^2}\Bigg\lvert_{r=r_1}=0,\quad r_1=\sqrt[6]{\frac{26}{7}}\sigma\\
    \end{aligned}
\end{equation*}
3.a.能带底:$k=0,E(0)=E_0-2E_1$\\
能带顶:$k=\frac{\pi}{a},E(\frac{\pi}{a})=E_0+2E_1$\\
b.$m^*=\frac{\hbar^2}{\frac{d^2E}{dk^2}}=\frac{1}{2a^2E_1cos(ka)}$\\
能带底:$m^*=\frac{\hbar^2}{2a^2E_1}$\\
能带顶:$m^*=-\frac{\hbar^2}{2a^2E_1}$\\
4.a.
\begin{equation*}
    \begin{aligned}
        & E=\frac{\hbar^2k^2}{2m}=\frac{2\hbar^2\pi^2}{ma^2}\\
    \end{aligned}
\end{equation*}
b.
\begin{equation*}
    \begin{aligned}
        & V(x)=-\frac{V_0}{4}(e^{i\frac{6\pi}{a}x}+e^{i\frac{2\pi}{a}x}+e^{-i\frac{2\pi}{a}x}+e^{-i\frac{6\pi}{a}x})\\
        & V_{\pm1}=V_{\pm3}=-\frac{V_0}{4}\\
    \end{aligned}
\end{equation*}
$\frac{\pi}{a}$处能隙为$\frac{V_0}{2}$,$\frac{2\pi}{a}$处能隙为$0$,$\frac{3\pi}{a}$处能隙为$\frac{V_0}{2}$。\\


\end{document}