\documentclass[UTF8]{ctexart}
\usepackage{subfigure}
\usepackage{caption}
\usepackage{amsmath,bm}
\usepackage{amssymb}
\usepackage{pifont}
\usepackage{geometry}
\usepackage{graphicx}
\usepackage{gensymb}
\usepackage{wrapfig}
\usepackage{titlesec}
\usepackage{float}
\usepackage{diagbox}
\usepackage{fancyhdr}
\usepackage{color}
\usepackage{bm}
\usepackage{siunitx}
\usepackage{ulem}
\usepackage{CJKulem}
\pagestyle{plain}
\geometry{a4paper,scale=0.8}
\CTEXsetup[format+={\raggedright}]{section} 
\title{操作系统2016期末}
\author{Deschain}
\titlespacing*{\section}
{0pt}{0pt}{0pt}
\titlespacing*{\subsection}
{0pt}{0pt}{0pt}
\titlespacing*{\paragraph}
{0pt}{0pt}{0pt}
\titlespacing*{\subparagraph}
{0pt}{0pt}{0pt}
\titleformat*{\section}{\normalsize}
\titleformat*{\subsection}{\normalsize}
\begin{document}
\maketitle
\section*{一、简答题}
\subsection*{1.}
(1)读文件;(2)高优先级进程抢占CPU;(3)CPU被分配给进程;(4)读文件完成
\subsection*{2.}
内核级线程由操作系统负责调度,用户级线程有拥护库函数负责调度。
\subsection*{3.}
通过利用CPU调度和虚拟内存技术,将物理计算机作为共享资源,让进程认为有自己的处理器和内存。
\subsection*{4.}
(1)互斥(2)非抢占(3)请求和保持(4)环路等待;
\subsection*{5.}
连续分配
\subsection*{6.}
4000位
\subsection*{7.}
(1)FCFS:公平,但性能较差。\\
(2)SSF:对中间磁道有利,但可能有进程处于饥饿状态。\\
(3)电梯算法:性能较好,对中间磁道有利,且没有进程会饿死。\\
(4)单向扫描算法:改进对中间磁道的偏好。在中负载或重负载时,性能比电梯算法好。\\
\subsection*{8.}
(1)字符设备:以字符为单位存储、传输信息,例如键盘。\\
(2)块设备:以数据块为单位存储、传输信息,例如磁盘。\\
\section*{二、综合应用题}
\subsection*{1.}
(a)$\frac{1}{4}(15+11+2+7)=8.75$\\
(b)$\frac{1}{4}(4+6+7+12)=7.25$\\
(c)$\frac{1}{4}(9+5+2+12)=7$\\
(d)$\frac{1}{4}(12+9+5+12)=9.5$\\
(e)$\frac{1}{4}(10+10+4+12)=9$\\
\subsection*{2.}
typedef Semaphore int;\\
Semaphore Inside = 10; //超市里还可以进的人数\\
Semaphore Cashier = 1; //收银员\\
void CustomerProcedure(void)\{\\
    P(Inside);\\
    V(Customer);\\
    购物;\\
    P(Cashier);\\
    结账;\\
    V(Inside);\\
\}\\
void CashierProcedure(void)\{\\
    P(Customer);\\
    结账;\\
    V(Cashier);\\
\}\\
\subsection*{3.}
(a)每页有250个整数,每250次缺页1次,共$\lceil\frac{256\time256}{250}\rceil=263$次\\
(b)每次访问都缺页,共$256^2=65536$次。
\subsection*{4.}
(a)10次\\
(b)10次\\
(c)对于FIFO算法,增加页框数不一定减少缺页次数\\
(d)W(k,t)=4\\
(e)W(k,t)=5\\








\end{document}