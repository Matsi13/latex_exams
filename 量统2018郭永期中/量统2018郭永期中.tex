\documentclass[UTF8]{ctexart}
\usepackage{subfigure}
\usepackage{caption}
\usepackage{amsmath}
\usepackage{amssymb}
\usepackage{pifont}
\usepackage{geometry}
\usepackage{graphicx}
\usepackage{gensymb}
\usepackage{wrapfig}
\usepackage{titlesec}
\usepackage{float}
\usepackage{diagbox}
\usepackage{fancyhdr}
\pagestyle{plain}
\geometry{a4paper,scale=0.8}
\CTEXsetup[format+={\raggedright}]{section} 
\title{量统2018郭永期中}
\author{Deschain}
\titlespacing*{\section}
{0pt}{0pt}{0pt}
\titlespacing*{\subsection}
{0pt}{0pt}{0pt}
\titlespacing*{\paragraph}
{0pt}{0pt}{0pt}
\titlespacing*{\subparagraph}
{0pt}{0pt}{0pt}
\titleformat*{\section}{\normalsize}
\begin{document}
\maketitle

\section*{一、(本题48分,每小题6分)简答题}
1.解释下列概念;(1)定态(2)量子隧穿效应(3)宇称\\
(1)体系的能量有确定值的状态。\\
(2)能量为E的粒子有一定概率穿透高度为$U_0(U_0>E)$的势垒。\\
(3)若$\psi(x)=\psi(-x)$,则是偶(正)宇称;若$\psi(x)=-\psi(-x)$,则是奇(负)宇称。\\
2.德布罗意关系阐明了微观粒子的粒子性$(E,p)$与波动性$(\nu,\lambda$或$\omega,k)$之间的关系,
用数学公式可将该关系表示为:\\
$E=\hbar\omega=h\nu,\quad\lambda=\frac{h}{p}$\\
3.已知升算符与降算符分别为$a^\dagger=\sqrt\frac{m\omega}{2\hbar}(x-\frac{i}{m\omega}p_x),
  a=\sqrt\frac{m\omega}{2\hbar}(x+\frac{i}{m\omega}p_x)$,其中$x$为坐标算符,$p_x$为动量算符,
计算对易关系$[a,a^\dagger]$。\\
\begin{equation*}
  [a,a^\dagger]=\frac{m\omega}{2\hbar}([x,x]+\frac{i}{m\omega}([\hat p_x,x]-[x,\hat p_x])
  +\frac{i}{m^2\omega^2}[\hat p_x,\hat p_x])
  =\frac{m\omega}{2\hbar}\times\frac{i}{m\omega}\times2i\hbar=1
\end{equation*}
4.在Franck-Herz实验中,用电子束撞击氢原子,使氢原子跃迁到第一激发态。设第一激发态电子能量的起伏为
$10^4eV$,根据量子力学原理估算氢原子第一激发态的寿命。\\
\begin{equation*}
  \Delta E\cdot\Delta t\geq\frac{\hbar}{2},\quad\Delta t\geq3.30\times10^{-10}s
\end{equation*}
5.一维谐振子的能级为(\quad\quad\quad\quad\quad),能级简并度为(\quad\quad\quad\quad\quad);
二维各向同性谐振子的能级为(\quad\quad\quad\quad\quad),能级简并度为(\quad\quad\quad\quad\quad);
三维各向同性谐振子的能级为(\quad\quad\quad\quad\quad),能级简并度为(\quad\quad\quad\quad\quad);\\
\begin{equation*}
  \begin{aligned}
     & (1)E_n=(n+\frac{1}{2})\hbar\omega\quad(2)1                    \\
     & (3)E_n=(n+1)\hbar\omega\quad(4)n+1                            \\
     & (5)E_n=(n+\frac{3}{2})\hbar\omega\quad(2)\frac{(n+1)(n+2)}{2}
  \end{aligned}
\end{equation*}
6.设力学量算符(厄米算符)$\hat F,\hat G$不对易,令$K=i(\hat F\hat G-\hat G\hat F)$。证明:
(1)$\hat K$的本征值是实数;(2)在任何态中,$\overline{F^2}+\overline{G^2}\geq\overline{K}$。\\
(1)
\begin{equation*}
  \hat K^\dagger=-i(\hat G^\dagger\hat F^\dagger-\hat F^\dagger\hat G^\dagger)
  =i(-\hat G\hat F+\hat F \hat G)=\hat K
\end{equation*}
$\therefore\hat K$是Hermite算符$\therefore\hat K$的本征值是实数。\\
(2)
\begin{equation*}
  \begin{aligned}
     & \hat A=\hat F+i\hat G,\quad\hat B=\hat F-i\hat G,\quad\hat A\hat B=\hat F^2+\hat G^2-\hat K^2 \\
     & \because <\psi\lvert\hat A\hat B\rvert\psi>=\lvert\hat B\lvert\psi>\rvert^2\geq0,\quad
    \therefore<\psi\lvert\hat F^2\hat G^2-\hat K\rvert\psi>\geq0,\quad
    \overline{F^2}+\overline{G^2}\geq\overline{K}                                                    \\
  \end{aligned}
\end{equation*}
7.$t=0$时,粒子的状态为$\psi(x)=Asin^2(kx)$,求此时动量的可能测值和相应的概率,并计算动量的平均值。\\
\begin{equation*}
  \begin{aligned}
     & \psi(x)=\frac{1}{6\sqrt{2\pi}}(e^{2ikx}+e^{-2ikx}+2)                                        \\
     & P(p_x=2k\hbar)=P(p_x=-2k\hbar)=\frac{1}{6},\quad P(p_x=0)=\frac{2}{3},\quad\overline{p}_x=0
  \end{aligned}
\end{equation*}
8.一量子体系由三个全同玻色子组成,玻色子之间无相互作用,玻色子只有可能的2个单粒子态$\varphi_1$和
$\varphi_2$,相应的能量为$\varepsilon_1$和$\varepsilon_2$,写出该量子体系所有可能态的波函数和能量。\\
\section*{二、(本题8分)设矩阵A和B满足$A^2=0,AA^\dagger+A^\dagger A=1,B=A^\dagger A$。\\
  (1)证明$B^2=B$;\\
  (2)在B表象中,求出A的矩阵表示(设B本征值无简并)。}
 (1)
\begin{equation*}
  \hat B^2=\hat A^\dagger\hat A\hat A^\dagger\hat A=\hat A^\dagger(1-\hat A^\dagger\hat A)\hat A
  =\hat A^\dagger\hat A=\hat B
\end{equation*}
(2)
\begin{equation*}
  \begin{aligned}
     & \because A^2=0,\quad B=A^\dagger A \\
     & \therefore A=
    \begin{bmatrix}
      0           & 0 \\
      e^{i\delta} & 0
    \end{bmatrix}
  \end{aligned}
\end{equation*}
\section*{三、(本题8分)证明并举1-2例说明定理:如果体系具有两个互相不对易的守恒量,那么体系的能级
  一般是简并的。}
1.对于氢原子,$\hat L_x,\hat L_y,\hat L_z$不对易,但都是守恒量,所以能级是简并的。\\
2.自由粒子的宇称和动量互相不对易,但都是守恒量,所以能级是简并的。
\section*{四、(本题8分)氢原子的波函数在球坐标系下写为$\psi(r,\theta,\varphi)=R(r)(e^{i\varphi}
    sin\theta+cos\theta)$,其中$R(r)$为径向函数。求解下列问题:\\
  (1)角动量平方$\vec L^2$的可能测量值和相应的概率;\\
  (2)角动量的$z$分量$L_z$的可能测量值、相应的概率及平均值。}
参考公式:
\begin{equation*}
  \begin{aligned}
     & Y_{00}(\theta,\varphi)=\frac{1}{\sqrt{4\pi}},\quad
    Y_{10}(\theta,\varphi)=\sqrt\frac{3}{4\pi}cos\theta,\quad
    Y_{1,\pm1}(\theta,\varphi)=\mp\sqrt\frac{3}{8\pi}sin\theta e^{\pm i\varphi}                    \\
     & Y_{20}(\theta,\varphi)=\sqrt\frac{5}{16\pi}(3cos^2\theta-1),\quad
    Y_{2,\pm1}(\theta,\varphi)=\mp\sqrt\frac{15}{8\pi}sin\theta cos\theta e^{\pm i\varphi}         \\
     & Y_{2,\pm2}(\theta,\varphi)=\sqrt\frac{15}{32\pi}sin^2\theta e^{\pm2i\varphi}                \\
    \newline
    \newline
     & \psi(r,\theta,\varphi)=\sqrt\frac{2}{3}R(r)Y_{11}(\theta,\varphi)
    +\sqrt\frac{1}{3}R(r)Y_{10}(\theta,\varphi)                                                    \\
     & (1)P(L^2=2\hbar^2)=1                                                                        \\
     & (2)P(L_z=\hbar)=\frac{2}{3},\quad P(L_z=0)=\frac{1}{3},\quad\overline{L_z}=\frac{2}{3}\hbar
  \end{aligned}
\end{equation*}
\section*{五、(本题12分)由两个自旋均为$\frac{\hbar}{2}$的非全同粒子组成的量子体系,设粒子间相互作用为
  $H=A\vec S_1\cdot\vec S_2$(不考虑轨道运动)。设初始状态$t=0$时,粒子1自旋“向上”$(S_{1z}=
    \frac{\hbar}{2})$,粒子2自旋“向下”$(S_{2z}=-\frac{\hbar}{2})$。求解下列问题:\\
  (1)分析并给出该系统的守恒量;\\
  (2)$t>0$时刻该两粒子体系的自旋波函数;\\
  (3)总自旋角动量$\vec S^2$和$S_z$的取值及概率。}
\begin{equation*}
  \begin{aligned}
     & (1)\{\hat H,\hat S,\hat S_1,\hat S_2\}                                      \\
     & (2)\hat H=\frac{A}{2}(\hat S^2-\hat S_1^2-\hat S_2^2)                       \\
     & j=0,m=0,E_1=-\frac{3}{4}A\hbar^2,\quad\psi_1=\lvert00>                      \\
     & j=1,m=1,E_2=\frac{1}{4}A\hbar^2,\quad\psi_{21}=\lvert11>                    \\
     & j=1,m=0,E_2=\frac{1}{4}A\hbar^2,\quad\psi_{22}=\lvert10>                    \\
     & j=1,m=-1,E_2=\frac{1}{4}A\hbar^2,\quad\psi_{23}=\lvert1-1>                  \\
     & \Psi(r,0)=\frac{1}{\sqrt2}(\lvert00>+\lvert10>)                             \\
     & \Psi(r,t)=\frac{1}{\sqrt2}\lvert00>e^{i\frac{3}{4}A\hbar t}
    +\frac{1}{\sqrt2}\lvert10>e^{-i\frac{1}{4}A\hbar t}                            \\
     & =\frac{1}{\sqrt2}e^{i\frac{1}{4}A\hbar t}(\lvert00>e^{i\frac{1}{2}A\hbar t}
    +\lvert10>e^{-i\frac{1}{2}A\hbar t})                                           \\
     & =E^{i\frac{1}{4}A\hbar t}(\lvert+->cos(\frac{A\hbar t}{2})
    -\lvert-+>isin(\frac{A\hbar t}{2}))                                            \\
     & (3)P(S^2=0)=P(S^2=2)=\frac{1}{2},\quad P(S_z=0)=1                           \\
  \end{aligned}
\end{equation*}
\section*{六、(本题16分)极性分子的转动可用平面转子描述,其哈密顿算符为$H_0=\frac{L_z^2}{2I}$。其中$L_z$
  代表角动量z轴投影,转轴z过分子的质心并与分子的电偶极矩$\vec D$垂直,$I$为分子的转动惯量。}
$[1]$写出它的能级和归一化能量本征函数,指出各能级的简并度。\\
$[2]$考虑加入一个微扰哈密顿量$H'=U_0 cos^2\varphi,$其中$U_0>0$是常数,$\varphi$为平面转子的转动角。
那么准确到微扰论的一级修正,求解:(1)基态能级有多大的移动?(2)第一激发态能级的一级近似。
\begin{equation*}
  \begin{aligned}
     & (1)E_n=\frac{n^2\hbar^2}{2I},\quad\psi_{n1}=\frac{1}{\sqrt{2\pi}}e^{i n\varphi},\quad
    \psi_{n2}=\frac{1}{\sqrt{2\pi}}e^{-i n\varphi},\quad f_0=1,\quad f_n=2(n\geq1)                       \\
     & (2)E_0^{(1)}={H'}_{00}=\int_0{2\pi}\frac{U_0}{2\pi}cos^2\varphi=\frac{U_0}{2}                     \\
     & {H'}_{11}=\int_0{2\pi}\frac{U_0}{2\pi}cos^2\varphi=\frac{U_0}{2}={H'}_{22}                        \\
     & {H'}_{12}=\int_0{2\pi}\frac{U_0}{2\pi}e^{-2i\varphi}cos^2\varphi d\varphi=\frac{U_0}{4}={H'}_{21} \\
     & \begin{vmatrix}
      \frac{U_0}{2}-E & \frac{U_0}{4}   \\
      \frac{U_0}{4}   & \frac{U_0}{2}-E
    \end{vmatrix}=0,\quad E_1^{(1)}=\frac{U_0}{4},\quad E_2^{(1)}=\frac{3U_0}{4}
  \end{aligned}
\end{equation*}

\end{document}