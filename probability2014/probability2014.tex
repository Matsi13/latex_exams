\documentclass[UTF8]{ctexart}
\usepackage{amsmath}
\usepackage{geometry}
\usepackage{graphicx}
\usepackage{gensymb}
\usepackage{wrapfig}
\usepackage{titlesec}
\usepackage{float}
\usepackage{fancyhdr}
\pagestyle{plain}
\geometry{a4paper,scale=0.8}
\CTEXsetup[format+={\raggedright}]{section} 
\title{2014年概率论期末试题}
\author{Deschain}
\titlespacing*{\section}
{0pt}{0pt}{0pt}
\titlespacing*{\subsection}
{0pt}{0pt}{0pt}
\titlespacing*{\paragraph}
{0pt}{0pt}{0pt}
\titlespacing*{\subparagraph}
{0pt}{0pt}{0pt}
\titleformat*{\section}{\normalsize}
\begin{document}
\maketitle
\section{市中心一家蛋糕店,一小时内进入这家店的顾客数服从参数为$\lambda$的泊松分布。蛋糕店内有K种蛋糕,每个顾客选取每种蛋糕的概率服从均匀分布。求一小时内被购买的蛋糕种类的期望。}
\paragraph{解答}
设第i种蛋糕有人买时$X_i=1$,反之$X_i=0$。设有N人进入蛋糕店。
\begin{equation*}
\begin{aligned}
&X=\sum_{i=1}^N{X_i}\\
&P(X_i=1\lvert N)=1-(1-\frac{1}{K})^N\\
&E(X\lvert N)=N-N(1-\frac{1}{K})^N\\
&E(X)=\sum_{n=0}^{\infty}{\frac{\lambda^n}{n!}e^{-\lambda}(n-n(1-\frac{1}{K})^n)}\\
&=\frac{1}{\lambda}-e^{-\frac{\lambda}{K}}\frac{1}{\lambda(1-\frac{1}{K})}
\end{aligned}
\end{equation*}
\section{设有长度为10的数列Xi,每个数的值的分布为U(0,1)。定义局部极大值为
Xi-1<Xi>Xi+1,设其数量为N。\\
(1)求E(N)。\\
(2)求$E(X_1N)$。}
\paragraph{(1)解答}
设$X_i$为局部最大值时$Y_i=1$,反之$Y_i=0$。
\begin{equation*}
\begin{aligned}
&Y_1=0,\quad Y_10=0\\
&P(Y_i=1)=\frac{1}{3},i=2,3,\cdots,9\\
&E(N)=E(\sum_{i=2}^9{Y_i})=\sum_{i=2}^9{E(Y_i)}=\frac{8}{3}
\end{aligned}
\end{equation*}
\paragraph{(2)解答}
\begin{equation*}
\begin{aligned}
&E(X_1N)=E(E(x_1N\lvert X_1))\\
&E(x_1N\lvert X_1)=x_1E(N\vert X_1)=x_1E(\sum_{i=3}^9{Y_i})+x_1E(Y_2\lvert X_1)\\
&=\frac{7}{3}x_1+x_1\int_{x_1}^{1}{dx_2}\int_0^{x_2}{dx_3}=\frac{17}{6}x_1-\frac{1}{2}x_1^3\\
&E(X_1N)=\int_0^1{(\frac{17}{6}x_1-\frac{1}{2}x_1^3)dx_1}=\frac{31}{24}
\end{aligned}
\end{equation*}
\section{}
设有联合密度函数\begin{equation*}
Z= f_{X,Y}(x,y)=\begin{cases}
c\lvert xy\rvert\quad\quad -1<x<1,0<y<\lvert x\rvert\\
0 \quad\quad else
\end{cases}
\end{equation*}
(1)求c\\
(2)求$P(x>y\lvert y<0.5)$。
\paragraph{(1)解答}
\[2\int_0^1{dx}\int_0^x{cxydy}=\frac{c}{4}=1\]
\[c=4\]
\paragraph{(2)解答}
\[P(x>y\lvert y<0.5)=\frac{P(x>y,y<0.5)}{P(y<0.5)}=\frac{P(x>0,y<0.5)}{P(y<0.5)}=\frac{1}{2}\]
\section{有两个灯泡,其寿命服从参数为$\lambda$的指数分布。同时测试两个灯泡,第一个灯泡使用$T_1$时间后损坏,之后第二个灯泡在$T_2$时间损坏。求$P(T_2>2T_1)$。}
\[P(T_2>2T_1)=\int_0^{\infty}{\lambda e^{-\lambda t_1}t_1}\int_{2t_1}^{\infty}{\lambda e^{-\lambda t_2}dt_2}=\frac{1}{3}\]
\section{有四个硬币,抛掷时正面朝上的概率均为p。刚开始四个硬币均处于反面朝
上的状态,从第一个硬币开始抛掷,若抛到反面继续抛;若抛到正面则抛掷下一个硬币。设抛掷次数为N,求P(N=6)。}
\paragraph{解答}
原事件相当于“前5次抛掷有且只有两次反面,第6次为正面”。
\[P=\frac{\tbinom{5}{2}}{2^5}\times\frac{1}{2}=\frac{5}{32}\]
\section{设随机变量$X\sim N(0,1)$,$Y\sim N(0,1)$,求$E(X^2-Y^2\lvert X^2+Y^2)$。}
\paragraph{解答}
\begin{equation*}
\begin{aligned}
E(X^2-Y^2\lvert X^2+Y^2)&=E(X^2\lvert X^2+Y^2)-E(Y^2\lvert X^2+Y^2)\\
&=E(X^2\lvert X^2+Y^2)-E(X^2\lvert X^2+Y^2)\\
&=0
\end{aligned}
\end{equation*}
\section{抛掷一枚骰子,若抛到6则停止抛掷,若抛到$i(1\leq i\leq 5)$则休息i分钟后继续抛掷。求抛掷时间X的期望。}
\paragraph{解答}
\begin{equation*}
\begin{aligned}
E(X)&=\frac{1}{6}(E(X)+5)+\frac{1}{6}(E(X)+4)+\frac{1}{6}(E(X)+3)+\frac{1}{6}(E(X)+2)+\frac{1}{6}(E(X)+1)\\
E(X)&=15
\end{aligned}
\end{equation*}
\section{在苏格拉底的花园中有9朵玫瑰,有一天,苏格拉底要柏拉图去花园中采一朵最漂亮的玫瑰,但有一个规则,不能走回头路,而且只能采一朵。柏拉图采取这样的策略:先观察前三朵玫瑰,之后若出现比前三朵都漂亮的玫瑰,则采下这朵玫瑰。求柏拉图采到的玫瑰是最美的那朵的概率。}
\paragraph{解答}
“柏拉图采到的玫瑰是最美的那朵”需要满足以下条件:\\
(1)第k朵玫瑰最美,$k\geq 4$;\\
(2)前$k-1$朵玫瑰中,最美的出现在前三朵中。\\
以第6朵玫瑰最美为例。第6朵玫瑰最美的概率为$\frac{1}{9}$,前5朵玫瑰中最美的出现在前三朵的概率为$\frac{3}{5}$。以此类推。
\begin{equation*}
\begin{aligned}
P&=\frac{1}{9}(1+\frac{3}{4}+\frac{3}{5}+\frac{3}{6}+\frac{3}{7}+\frac{3}{8})\\&=\frac{341}{840}
\end{aligned}
\end{equation*}
\section{设有N个点,每两个点间有线相连的概率为p,若三个点每两个点间均有线相连,则称这三个点构成一个三角形。求三角形数量的期望和方差。}
\paragraph{解答}

\end{document}