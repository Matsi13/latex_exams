\documentclass[UTF8]{ctexart}
\usepackage{subfigure}
\usepackage{caption}
\usepackage{amsmath,bm}
\usepackage{amssymb}
\usepackage{pifont}
\usepackage{geometry}
\usepackage{graphicx}
\usepackage{gensymb}
\usepackage{wrapfig}
\usepackage{titlesec}
\usepackage{float}
\usepackage{diagbox}
\usepackage{fancyhdr}
\usepackage{color}
\usepackage{bm}
\pagestyle{plain}
\geometry{a4paper,scale=0.8}
\CTEXsetup[format+={\raggedright}]{section} 
\title{固物2015期中A卷}
\author{Deschain}
\titlespacing*{\section}
{0pt}{0pt}{0pt}
\titlespacing*{\subsection}
{0pt}{0pt}{0pt}
\titlespacing*{\paragraph}
{0pt}{0pt}{0pt}
\titlespacing*{\subparagraph}
{0pt}{0pt}{0pt}
\titleformat*{\section}{\normalsize}
\begin{document}
\maketitle
\section*{\bfseries 1.填空题(每空1分,共34分)}
(1)布拉菲点阵中,能够完全平移覆盖的最小单元称为\underline{\makebox[3em]{}},每个单元含有
\underline{\makebox[2em]{}}个格点。\\
(2)Li、Na、K等碱金属的晶格属于\underline{\makebox[6em]{}}晶格,其中格点的面密度最大的晶面系的密勒指数为
\underline{\makebox[4em]{}}。设晶格常数为$a$,则该晶面系相邻晶面之间的面间距为\underline{\makebox[4em]{}}。\\
(3)晶体中面间距为$d$的一组晶面对应倒格子空间中的一个\underline{\makebox[3em]{}},这组晶面对应的倒格矢长
度为\underline{\makebox[3em]{}};若使用波长为$\lambda$的X射线对这一晶体进行衍射,则与这组平行晶面对应的一
级衍射峰的衍射角为\underline{\makebox[9em]{}}。\\
(4)晶体中,空位与间隙原子都是典型的\underline{\makebox[2em]{}}缺陷。除此之外,晶体中的缺陷按其几何特征还
可以分为\underline{\makebox[2em]{}}缺陷和\underline{\makebox[2em]{}}缺陷。\\
(5)金刚石材料中C-C键结合的方式是典型的\underline{\makebox[6em]{}},其电离度为\underline{\makebox[2em]{}};
除了这种结合方式外,固体的结合方式还有\underline{\makebox[6em]{}},\underline{\makebox[6em]{}},
\underline{\makebox[9em]{}}。
(6)在索末菲自由电子模型中,对于一种二维材料,其总面积为$S$,则其对应的$k$空间的点阵密度为
\underline{\makebox[4em]{}},能量标度下自由电子的能态密度为\underline{\makebox[4em]{}}。\\













\newpage
\section*{\bfseries 1.填空题}
(1)\ding{172}原胞\makebox[2em]{}
\ding{173}1\\
(2)\ding{172}体心立方\makebox[2em]{}
\ding{173}(100)或(101)或(011)\makebox[2em]{}
\ding{174}$\frac{\sqrt2}{2}a$\\
(3)\ding{172}点/格点/格矢\makebox[2em]{}
\ding{173}$\frac{2\pi}{d}$\makebox[2em]{}
\ding{174}$2arcsin(\frac{\lambda}{2d})$\\
(4)\ding{172}点\makebox[2em]{}
\ding{173}线\makebox[2em]{}
\ding{174}面\\
(5)\ding{172}共价键(共价结合)\makebox[2em]{}
\ding{173}0\makebox[2em]{}
\ding{174}离子键(离子结合)\makebox[2em]{}
\ding{175}金属键(金属性结合)\makebox[2em]{}
\ding{176}范德瓦耳斯力(范德瓦尔斯结合)\\
(6)\ding{172}$\frac{S}{4\pi^2}$\makebox[2em]{}
\ding{173}$\frac{mS}{\pi\hbar^2}$\\


\end{document}