\documentclass[UTF8]{ctexart}
\usepackage{amsmath}
\usepackage{geometry}
\usepackage{graphicx}
\usepackage{gensymb}
\usepackage{wrapfig}
\usepackage{titlesec}
\usepackage{float}
\usepackage{diagbox}
\usepackage{fancyhdr}
\pagestyle{plain}
\geometry{a4paper,scale=0.8}
\CTEXsetup[format+={\raggedright}]{section} 
\title{2019年概率论期末试题及解答}
\author{Deschain}
\titlespacing*{\section}
{0pt}{0pt}{0pt}
\titlespacing*{\subsection}
{0pt}{0pt}{0pt}
\titlespacing*{\paragraph}
{0pt}{0pt}{0pt}
\titlespacing*{\subparagraph}
{0pt}{0pt}{0pt}
\titleformat*{\section}{\normalsize}
\begin{document}
\maketitle
\section{将52张扑克牌洗匀后,一张一张翻开,直到翻出红色的A为止。请计算所需翻牌次数的均值。}
\paragraph{解答}
\begin{equation*}
\begin{aligned}
&P(X=k)=\frac{1}{52}\\
&E(X)=\sum\limits_{k=1}^{52}\frac{k}{52}=26.5\\
\end{aligned}
\end{equation*}
\section{直角坐标系中,三角形ABC的三个顶点分别为A(-1,0),B(1,0)和C(0,1)。在该三角形的内部随机取一个点,设其坐标为(X,Y),请计算(X,Y)的联合概率密度和各自边缘密度,并判断X与Y是否相互独立?}
\paragraph{解答}
\begin{equation*}
\begin{aligned}
&f_{XY}(x,y)=1,-1<x<1,0<y<1-\lvert x\rvert\\
&f_X(x)=\int_0^{1-\lvert x\rvert}f_{XY}(x,y)dy=1-\lvert x\rvert,-1<x<1\\
&f_Y(y)=\int_{y-1}^{1-y}f_{XY}(x,y)dx=2-2y,0<y<1\\
\end{aligned}
\end{equation*}
\\
X,Y不独立
\section{假定有A和B两个彩票站,所售彩票的中奖概率分别为a和b,a>b,彩票价格分别为2元和1元。夫妻两人分别在两个彩票站买彩票,每个月买一张,两人都会在自己中奖后停止购买并结算花费。请计算当a和b满足什么条件时,丈夫花费钱数的均值大于妻子花费钱数的均值?}
\paragraph{解答}
设丈夫购买X张,妻子购买Y张。
\begin{equation*}
\begin{aligned}
&X\sim Ge(a),Y\sim Ge(b)\\
&E(X)=\frac{1}{a},E(Y)=\frac{1}{b}\\
\end{aligned}
\end{equation*}
\\
a<2b时,丈夫花费钱数的均值大于妻子花费钱数的均值。
\section{考虑独立的随机变量:$X_1,X_2\sim N(0,\sigma_1^2),Y_1,Y_2\sim N(0,\sigma_2^2)$,如果$A=X_1^2+X_2^2,B=Y_1^2+Y_2^2$。请计算(这里t为确定性参数)$P(A<B\lvert A>t)+P(A>B\lvert B>t$。}
\paragraph{解答}
\begin{equation*}
\begin{aligned}
&\begin{cases}
X_1=\rho cos\theta\\
X_2=\rho sin\theta\\
\end{cases}
\lvert \frac{\partial(x_1,x_2)}{\partial(\rho,\theta)}\rvert=\rho\\
&f_{\rho,\theta}(\rho,\theta)=f_{X_1,X_2}(\rho cos\theta,\rho sin\theta)\rho=\frac{\rho}{1\pi\sigma_1^2}e^{-\frac{\rho^2}{2\sigma_1^2}},\rho>0,0<\theta<2\pi\\
&f_\rho(\rho)=\int_0^{2\pi}f_{\rho,\theta}(\rho,\theta)d\theta=\frac{\rho}{\sigma_1^2}e^{-\frac{\rho^2}{2\sigma_1^2}}\\
&A=\rho^2,f_A(a)=\frac{1}{2\sigma_1^2}e^{-\frac{a}{2\sigma_1^2}},F_A(a)=1-e^{-\frac{a}{2\sigma_1^2}}\\
&f_B(b)=\frac{1}{2\sigma_2^2}e^{-\frac{b}{2\sigma_2^2}},F_B(b)=1-e^{-\frac{b}{2\sigma_2^2}}\\
&f_{AB}(a,b)=\frac{1}{4\sigma_1^2\sigma_2^2}e^{-\frac{a}{2\sigma_1^2}-\frac{b}{2\sigma_2^2}}\\
&P(A>t)=\int_t^{+\infty}\frac{1}{2\sigma_1^2}e^{-\frac{a}{2\sigma_1^2}}=e^{-\frac{t}{2\sigma_1^2}}\\
&P(B>t)=\int_t^{+\infty}\frac{1}{2\sigma_2^2}e^{-\frac{b}{2\sigma_2^2}}=e^{-\frac{t}{2\sigma_2^2}}\\
&P(A<B\lvert A>t)=\frac{P(t<A<B)}{P(A>t)}=\frac{\int_t^{+\infty}da\int_a^{+\infty}f_{AB}(a,b)db}{e^{-\frac{t}{2\sigma_1^2}}}=\frac{\sigma_2^2}{\sigma_1^2+\sigma_2^2}e^{-\frac{t}{2\sigma_2^2}}\\
&P(A<B\lvert B>t)=\frac{P(A<B,B>t)}{P(B>t)}=\frac{\int_t^{+\infty}db\int_0^bf_{AB}(a,b)da}{e^{-\frac{t}{2\sigma_2^2}}}=1-\frac{\sigma_1^2}{\sigma_1^2+\sigma_2^2}e^{-\frac{t}{2\sigma_1^2}}\\
&P(A<B\lvert A>t)+P(A<B\lvert B>t)=1-\frac{\sigma_1^2}{\sigma_1^2+\sigma_2^2}e^{-\frac{t}{2\sigma_1^2}}+\frac{\sigma_2^2}{\sigma_1^2+\sigma_2^2}e^{-\frac{t}{2\sigma_2^2}}\\
\end{aligned}
\end{equation*}
\section{从0-499这500个整数中随机取一个数X,并计算其各位数字之和S(例如,245的各位数字之和即为S=2+4+5=11),请计算S的均值。}
\paragraph{解答}
\begin{equation*}
S=\frac{1}{500}(45\times\frac{500}{10}+45\times\frac{500}{100}\times10+(1+2+3+4)\times100)=11\\
\end{equation*}
\section{设非负随机变量$X_1,X_2$独立同分布,设U在区间$[0,t]$上均匀分布,且与$X_1,X_2$相互独立,试计算$P(U<X_1\lvert X_1+X_2=t)=?$}
\paragraph{解答[法一]}
设$X_1$和$X_2$的密度函数为$f_X(x),Y=X_1+X_2$\\
\begin{equation*}
\begin{aligned}
&f_Y(y)=\int_0^yf_X(y-x)f_X(x)dx\\
&f_{X_1X_2}(x_1,x_2)=f_X(x_1)f_X(x_2)\\
&\frac{\partial(y,x_1)}{\partial(x_1,x_2)}=1\\
&f_{X_1Y}(x_1,y)=f_X(x_1)f_X(y-x_1)\\
&f_{X_1\lvert Y}(x_1\lvert y)=\frac{f_X(x_1)f_X(y-x_1)}{\int_0^tf_X(x)f_X(t-x)dx},0<x_1<t\\
&X_1=x_1,P(U<x_1\lvert X_1<x_1,X_2=t-x_1)=P(U<x_1)=\frac{x_1}{t}\\
&P(U<X_1\lvert X_1+X_2=t)=\int_0^tP(U<X_1\lvert X_1=x_1,X_2=t-x_1)f_{X_1\lvert t}(x\lvert t)=\frac{\int_0^t\frac{x_1}{t}f_X(x_1)f_X(t-x_1)dx_1}{\int_0^tf_X(t-x)f_X(x)dx}\\
&\int_0^{\frac{t}{2}}x_1f(x_1)f_X(t-x_1)dx_1=\int_{\frac{t}{2}}^t(t-x_1)f_X(x_1)f_X(t-x_1)dx_1\\
&\int_0^tx_1f(x_1)f_X(t-x_1)dx_1=\int_{\frac{t}{2}}^ttf_X(x_1)f_X(t-x_1)dx_1\\
&P(U<X_1\lvert X_1+X_2=t)=\frac{\int_{\frac{t}{2}}^tf_X(x_1)f_X(t-x_1)dx_1}{\int_0^tf(x_1)f_X(t-x_1)dx_1}=\frac{1}{2}\\
\end{aligned}
\end{equation*}
\paragraph{解答[法二]}
U与$X_1,X_2$独立,关于U的概率仅与区间长度有关。
\begin{equation*}
\begin{aligned}
&P(U<X_1\lvert X_1+X_2=t)=P(t-X_1<U<t\lvert X_1+X_2=t)\\
&=P(X_2<U<t\lvert X_1+X_2=t)=P(U>X_1\lvert X_1+X_2=t)\\
&P(U<X_1\lvert X_1+X_2=t)+P(U>X_1\lvert X_1+X_2=t)=1\\
&P(U<X_1\lvert X_1+X_2=t)=\frac{1}{2}\\
\end{aligned}
\end{equation*}
\section{设随机变量X,Y相互独立,且$X\sim Exp(\lambda),Y\sim Exp(\mu)$设$Z=min(X,Y)$,$W=\begin{cases}
1,\quad if\quad Z=X\\
0,\quad if\quad Z=Y\\
\end{cases}$\\
试求Z与W的相关系数?}
\paragraph{解答}
\begin{equation*}
\begin{aligned}
&f_Z(z)=(\lambda+\mu)e^{-(\lambda+\mu)z}\\
&P(W=1)=\frac{\lambda}{\lambda+\mu},P(W=0)=\frac{\mu}{\lambda+\mu}\\
&Y=WZ,Y=\begin{cases}
X,\quad X<Y\\
0,\quad X>Y\\
\end{cases}
\\
&E(Y)=\int_0^{+\infty}dy\int_0^y\lambda\mu xe^{-\lambda x-\mu y}dx=(\frac{\lambda}{\lambda+\mu})^2\\
&E(Z)=\frac{1}{\lambda+\mu}\\
&E(W)=\frac{\lambda}{\lambda+\mu}\\
&E(ZW)-E(Z)E(W)=Cov(Z,W)=0\\
&Corr(Z,W)=0\\
\end{aligned}
\end{equation*}
\section{设随机变量X和Y相互独立,且均服从参数为1的指数分布。定义$U=X+Y$,$V=\frac{X}{X+Y}$,试写出U和V的联合分布,并计算$E(V\lvert U)$。}
\paragraph{解答}
\begin{equation*}
\begin{aligned}
&X=UV,Y=U-UV,\frac{\partial(x,y)}{\partial(u,v)}=
\begin{vmatrix}
v&u\\
1-v&-u
\end{vmatrix}
=\lvert u\rvert=u\\
&f_{UV}(u,v)=ue^{-uv}e^{-(u-uv)}=ue^{-u},u>0,0<v<1\\
&f_V(v)=\int_0^{+\infty}ue^{-u}du=1\\
&f_U(u)=\int_0^1ue^{-u}dv=ue^{-u}\\
&f_{V\lvert U}=\frac{f_{UV}(u,v)}{f_U(u)}=1\\
&E(V\lvert U)=\int_0^1vdv=\frac{1}{2}\\
\end{aligned}
\end{equation*}
\section{在边长为2的正方形区域内随机选择一个点,计算该点到区域边界的最短距离的均值与方差。}
\paragraph{解答}
将正方形沿两条中线和两条对角线切分成8个区域,由对称性,在求密度函数时可用下半侧的左起第二个代表其他区域。设距离为Z。
\begin{equation*}
\begin{aligned}
&f_{XY}(x,y)=\frac{1}{4},0<y<x<1\\
&Z=Y\\
&f_Y(y)=\int_y^1\frac{1}{4}dx=\frac{1-y}{4}\\
&f_Z(z)=8f_Y(z)=2-2z,0\leq z\leq1\\
&E(Z)=\frac{1}{3},Var(Z)=\frac{1}{18}\\
\end{aligned}
\end{equation*}
\section{连续抛掷硬币,直到出现连续的一个正面一个反面(或一个反面一个正面)位置,计算已经完成的抛掷中,正面向上数目的期望与方差。}
\paragraph{解答}
两种情况:(1)N正+1反(2)N反+1正,(1)和(2)是等概率的。设正面数为Y。\\
注:本题中级数求和的结果直接给出,详细证明来自几何分布的均值与方差。
\begin{equation*}
\begin{aligned}
&P(Y=1)=\lim_{n\to\infty}\frac{1}{4}+(\frac{1}{4}+\frac{1}{8}+\cdots+\frac{1}{2^n})=\frac{3}{4}\\
&P(Y=k)=\frac{1}{2^{k+1}}\quad\quad(k\geq2)\\
&E(Y)=\frac{3}{4}+\sum\limits_{k=2}^{+\infty}\frac{k}{2^{k+1}}=\frac{3}{2}\\
&E(Y^2)=\frac{3}{4}+\sum\limits_{k=2}^{\infty}\frac{k^2}{2^{k+1}}=\frac{7}{2}\\
&Var(Y)=E(Y^2)-E^2(Y)=\frac{5}{4}\\
\end{aligned}
\end{equation*}
\end{document}