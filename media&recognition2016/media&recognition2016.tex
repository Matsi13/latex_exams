\documentclass[UTF8]{ctexart}
\usepackage{amsmath}
\usepackage{geometry}
\usepackage{diagbox}
\usepackage{bm}
\geometry{a4paper,scale=0.8}
\title{2016年媒体与认知期末试题}
\author{Deschain}
\begin{document}
\maketitle
\section{简答题}
\subsection{简述主成分分析(PCA)与线性鉴别分析(LDA)的原理}
\paragraph{解答}
\subparagraph{PCA}
最小均方误差准则下保留原始数据信息。无监督,取数据投影方差最大的方向。
具体方法:对于n维样本,减去平均值变为零均值样本,协方差矩阵记为$\Sigma$。求出$\Sigma$的前d个最大的特征值和对应的特征向量。用这些特征向量组成的$d\times n$维矩阵将样本降至d维。
\subparagraph{LDA}
最小均方误差准则下区分多类数据。有监督,取分类性能最好的投影方向。
具体方法:假设有n个p维样本$x_i$,标签为$y_i$,对应C个类别$w_j$,每一类样本数量为$N_j$,均值为$\mu_j$, 样本总体均值为$\mu$。类内散度矩阵$S_w=\sum_{i=1}^C{\frac{N_j}{N}\sum_{k=1}^{N_j}{\frac{(x_ki-\mu_i)(x_ki-\mu_i)^T}{N_j}}}$表示样本点x围绕各类均值的散布情况。类间散度矩阵$S_b=\sum_{i=1}^C{\frac{n_i}{N}(\mu_i-\mu)(\mu_i-\mu)^T}$总散度矩阵$S_T=S_w+S_b$,$S_T=\sum_{i=1}^n{\frac{(x_i-\mu)(x_i-\mu)^T}{N}}$。对$S_w^{-1}S_b$做广义特征值分解,得到的最大的几个特征值、特征向量用于降维。
\subsection{简述记忆的主要类型}
\paragraph{解答}
\subparagraph{感觉记忆}
(1)定义:感觉记忆又可称为瞬时记忆,是一种信息储存时间以毫秒或秒计的记忆。心理学家假设每一种感觉通道都有一种感觉记忆,每一种感觉记忆都能将感觉刺激的物理特征的精确表征保持几秒钟或更短的时间。感觉记忆是记忆系统的开始阶段,它是一种原始的感觉形式,是记忆系统在对外界信息进行进一步加工之前的暂时登记。
(2)特点
①存储时间非常短:图像记忆在几百毫秒内,声像记忆可达4秒。信息加工只是初步的(但可以进行信息整合)。基本是按照刺激的物理特点进行编码,是外界刺激的真实复本。
②记忆容量非常大:图像记忆在9-20个项目内,声像记忆容量小于图像记忆。但只有一部分信息会进入到高一级的短时记忆中。
③记忆过程是无意识的自动化的,人无法控制。
\subparagraph{短时记忆}
一种信息存储时间为一分钟以内(约15-30秒)的记忆。又可被称为电话号码式记忆。是个体对刺激信息进行加工、编码、短暂保持和容量有限的记忆。
(1)记忆的组织是一种表列等级结构,从短时记忆向长时记忆存入一项需要5-10秒。
(2)短时记忆信息的编码:在记忆系统中对信息进行转换,使之获得适合于记忆系统的形式,经过编码所产生的具体的信息形式为代码。
(3)短时记忆信息的存储、遗忘:短时记忆的保持时间为15-30秒,内容会随时间逐渐减少。
\subparagraph{长时记忆}
将信息保持存储时间在一分钟以上的记忆。存储信息的数量随时间的推移而逐渐下降。受知识和经验差异的影响,人们存储的经验可能会发生不同程度的变化,例如扭曲和错觉。
\section{名词解释}
\subsection{认知}
\paragraph{解答}
指人认识客观世界事物的过程,是对作用于人的感觉器官的客观世界事物进行加工处理的过程。人的认知需要对进入视觉、听觉等感知器官的信息进行加工和处理,才能被人的大脑理解。人的认知包括感知、注意、记忆、学习、思维、意识、情绪等,可分为视觉认知、听觉认知、语言认知。
\subsection{简单感受野}
\paragraph{解答}
对大面积弥散光刺激没有反应,而对有一定方向或朝向的条纹刺激有强烈反应。若该刺激物的方向偏离该细胞“偏爱”的最优方位,则细胞反应停止或骤减。同时,它们对该类视觉刺激的位置和空间频率也表现出了明显的选择性。
\subsection{虚拟现实}
\paragraph{解答}
一种能够创建和体验虚拟环境的,由计算机生成的,提供多种感官刺激的自然人机交互系统。主要思想:人的感知系统接收虚拟真实场景->临场感。人所看到的场景随视点的变化而实时变化。三个特征:沉浸、交互、幻想。
\section{计算题}
\subsection{}
\paragraph{}
考虑一个神经元,它采用Sigmoid函数作为激活函数,$s(z)=\frac{1}{1+e^{-x}}$,在本题计算中可采用$e\approx 2.7$,$e^{-3}\approx 0.05$。神经元输入为$X=(x_0,x_1,\cdots,x_n)^T$,n为输入数据的个数,偏置量$x_0=1$;权值向量为$W=(w_0,w_1,\cdots,w_n)^T$;$z=W_Tx$,神经元输出为$y=s(z)$,样本对应的类别真值为g;误差函数为$E=\frac{1}{2}(g-y)^2$。
\subparagraph{(1)}
请写出权值更新公式(学习率为$\lambda$)。
\begin{equation*}
\begin{aligned}
&\frac{\partial E}{\partial y}=y-g \\
&\frac{\partial y}{\partial z}=y(1-y) \\
&\frac{\partial z}{\partial \boldsymbol W}=\boldsymbol x \\
&\boldsymbol W_{new} = \boldsymbol W -\lambda (y-g)(1-y)y\boldsymbol x
\end{aligned}
\end{equation*}
\subparagraph{(2)}
神经元输入为$X=(x_0,x_1,x_2)^T=(1,2,0.5)^T$,样本对应的类别真值$y=1$,权值$W=(w_0,w_1,w_2)^T=(0.5,1,1)^T$,即$w_0(1)=0.5$,$w_1(1)=1$,$w_2(1)=1$。设学习率$\lambda =0.1$,请计算误差反向传播之后,权值更新一次得到的数值$w_0(2)$,$w_1(2)$,$w_2(2)$。
\begin{equation*}
\begin{aligned}
z&=\boldsymbol {W^Tx} =3\\
y&=1+e^{-3}\\
g&=1\\
\boldsymbol W_{new} &=(0.50021425,1.0004285,1.000107125)
\end{aligned}
\end{equation*}
\subsection{}
\paragraph{}
在X空间中给定四个训练样本,正样本为$X_1=(0,0)^T$,$X_2=(2,2)^T$,负样本为$X_3=(h,1)^T$,$X_4=(0,3)^T$,其中$h\geq 0$为常数。
\subparagraph{(1)}
请给出h的取值范围,使得训练样本为线性可分。
\[h<1\]
\subparagraph{(2)}
在线性可分情形下,请写出由最大间隔分类界面得到的间隔表达式,以h作为参数。
\[\frac{2x}{1-h}-\frac{2y}{1-h}+1=0\]
\subparagraph{(3)}
假设我们只能观测到第2个坐标(纵轴),即正样本为$X_1=0$,$X_2=2$,对应类别标号$y_1=-1$,$y_4=-1$。若四个样本均为支持向量,支持向量机采用多项式核函数$K(X_i,X_j)=(1+X_i^TX_j)^p$,请写出决策函数$g(x)=\sum{j=1}^4{y_j\alpha_jK(x,X_j)+b}$的表达式(不用求出$\alpha_j$和b);通过观察数据分布情况,说明多项式核函数采用的最低阶数p的取值(p为正整数)。
\[
g(x)=\alpha_1+\alpha_2(1+2x)^p-\alpha_3(1+x)^p-\alpha_4(1+3x)^p+b, \quad p\geq 2
\]
\subsection{}
\paragraph{}
对于一个隐含马尔可夫模型,在$t(t=1,2,\cdots,n)$时刻,状态$Q_i=S_i$,$i=1,2,3$。观测为$O_t\in \{x,y,z\}$,模型参数$\lambda=(\pi,A,B)$为:
初始时刻为$t=1$,初始分布为$\pi_1=1$,$\pi_2=0$,$\pi_3=0$,
状态转移矩阵$\begin{bmatrix} 
\frac{1}{2} & \frac{1}{4} & \frac{1}{4}\\
0 & \frac{1}{2} & \frac{1}{2} \\
0 & 0 & 1
\end{bmatrix}$,观测概率矩阵$\begin{bmatrix}
\frac{1}{2} & \frac{1}{2} & 0\\
\frac{1}{2} & 0 & \frac{1}{2} \\
0 & \frac{1}{2} & \frac{1}{2}
\end{bmatrix}$
\subparagraph{(1)}
请计算时刻$t=5$,状态$Q_5$为$S_3$的概率$P(Q_5=S_3)$。

解答:
\begin{tabular}{|l|l|l|l|l|l|}
\hline
\quad & 1 & 2 & 3 & 4 & 5 \\
\hline
$S_1$ & 1 & $\frac{1}{2}$ & $\frac{1}{4}$ & $\frac{1}{8}$ & \quad \\
\hline
$S_2$ & 0 & $\frac{1}{4}$ & $\frac{1}{4}$ & $\frac{3}{16}$ & \quad \\
\hline
$S_3$ & 0 & $\frac{1}{4}$ & $\frac{1}{2}$ & $\frac{11}{16}$ & $\frac{13}{16}$\\
\hline
\end{tabular}
\subparagraph{(2)}
假设观测为x,x,y,z,y,请计算前向变量$\alpha_t(i)=P(O_1,\cdots,O_5,Q_t=i)$,$t=1,2,\cdots,5$,$i=1,2,3$。
\begin{tabular}{|l|l|l|l|l|l|}
\hline
t & 1 & 2 & 3 & 4 & 5 \\
\hline
$\alpha_t(1)$ & $\frac{1}{2}$ & $\frac{1}{8}$ & $\frac{1}{32}$ & 0 & 0 \\
\hline
$\alpha_t(2)$ & 0 & $\frac{1}{16}$ & 0 & $\frac{1}{256}$ & 0\\
\hline
$\alpha_t(3)$ & 0 & 0 & $\frac{1}{32}$ & $\frac{5}{256}$ & $\frac{11}{1024}$ \\
\hline
\end{tabular}
\section{论述题}
\paragraph{}
今年3月,围棋机器人Alpha Go以4:1的大比分战胜了围棋世界冠军李世石,成为人工智能技术发展的一个里程碑。今年初夏,一则“机器人将迎战2017年文科高考”的新闻又引起了社会上的关注。明年高考期间,“高考机器人”将单独在一个关闭外部网络的房间内,由专业公证人员监考,输入试卷电子版来“读题”,通过内部服务器的计算,最终由打印机输出其答案,与全国文科高考生同时考试,同时交卷。没有准备让机器人参加理科高考,是由于理科试卷中的图形题解析难度较大。让机器人参加高考,只是为了检验人工智能发展的水平,并以此推动各项技术的发展。目前,类似高考机器人的人工智能技术已经可以自动评测试卷,对答案正误、错误原因进行智能判定,应用于辅助教学。在应用中发现,学生的成绩提升呈现一定规律,即班级里排名靠前的学生进步最大,中游学生次之,排名靠后的学生仍然原地踏步。结合媒体与认知课程内容,请论述:
\subparagraph{1}
研制“高考机器人”需要哪些关键技术?\\
解答:单个文字识别(基于MLP),文字序列识别(基于CNN或RNN),语义分析(基于CNN或RNN),语篇输出能力(基于RNN)。
\subparagraph{2}
人工智能技术的应用对人的认知有何影响?\\
解答:提升感知能力,提升认知能力(大数据),取代人的脑力劳动。\\
注:论述题并未在近年的考试中出现,不是重点。
\end{document}
