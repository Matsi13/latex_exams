\documentclass[UTF8]{ctexart}
\usepackage{amsmath}
\usepackage{geometry}
\usepackage{diagbox}
\geometry{a4paper,scale=0.8}
\title{2015年概率论期末试题}
\author{Deschain}
\begin{document}
\maketitle
\paragraph{}
1.一个粒子在二维空间中做随机运动,方向可以是上,下,左,右四种,且各方向等概,每一步的步长为1。假定从原点开始,每一秒运动一次,以第n秒时粒子所在位置与原点的距离为半径画圆,请计算这个圆面积的均值。
\subparagraph{[法一]}
设n秒时,共左右移动X次,上下移动$n-X$次,则$X\sim B(X,\frac{1}{2})$。此时水平方向的位移的平方
\begin{equation*}
\begin{aligned}
Z&=(X-2Y)^2=X^2-4XY+4Y^2 \\
E(Z\lvert X)&= X^2-4X*E(Y\lvert X)+4E(Y^2\lvert X) \\
&= X^2-4X*\frac{X}{2}+4[Var(Y\lvert X)+E^2(Y \lvert X)]
\end{aligned}
\end{equation*}
同理垂直方向位移的平方
\[ E(V\lvert X)=n-X\]
设面积为S,则
\[E(S)=\pi (E(V)+E(Z))=n\pi \]
\subparagraph{[法二]}
设有a次上下移动,b次左右移动,$a+b=n$。
记第k次上下移动量为$I_k(I_1,I_2,\cdots ,I_n =\pm 1)$
\begin{equation*}
\begin{aligned}
X&=\sum_{k=1}^a I_k \\
E(X^2)&=E[(\sum_{k=1}^a I_k)^2] \\
&=E(\sum_{k=1}^a I_k^2)+E(\sum_{m\neq n}^{}I_m I_n) \\
&=\sum_{k=1}^a E(I_k^2) + sum_{m\neq n}^{}E(I_m)E(I_n) \\
&=a\times 1+0 \\
&=a
\end{aligned}
\end{equation*}
同理$E(Y^2)=b$
\[ E(S)=\pi E(X^2+Y^2)=\pi (a+b)=n\pi \]
\paragraph{}
2.设枪库里有两种枪,其中一种经过校正,命中率为p,另外一种尚未校正,命中率为q。考虑两种射击方式:从枪库里任取一支枪,射击一次,然后放回,如此连续两次;或者是从枪库里任取一支枪,独立射击两次。请问哪一种方式有更高的两次均命中的概率?
\begin{equation*}
\begin{aligned}
&P1=(\frac{p+q}{2})^2 \\
&P2=\frac{p^2+q^2}{2}
\end{aligned}
\end{equation*}
$P2\geq P1$(当且仅当$p=q$时取“=”)
\paragraph{}
3.随机变量X,Y相互独立,分别服从参数为$\lambda, \mu$的指数分布。设
\begin{equation*}
Z= \begin{cases}
4X+1,\quad X>Y \\
5Y-2,\quad X<Y
\end{cases}
\end{equation*}
请计算Z的概率密度与均值。\\
(1)$X>Y$
\begin{equation*}
\begin{aligned}
&f_{X_1}(x)=\int_0^x{\lambda e^{-\lambda x}\mu e^{-\mu y}dy}=\lambda e^{-\lambda x}(1-e^{-\mu x}) ,x>0\\
&Z_1=4X+1 \\
&f_{Z_1}(z)=\frac{1}{4}f_{X_1}(\frac{z-1}{4}) \\
&f_{Z_1}(z)=\frac{\lambda}{4}e^{-\frac{\lambda (z-1)}{4}}(1-e^{-\frac{\mu (z-1)}{4}}) , z \geq 1
\end{aligned}
\end{equation*}
(2)$X<Y$
\begin{equation*}
\begin{aligned}
&f_{Y_2}(y)=\int_0^y{\lambda e^{-\lambda x}\mu e^{-\mu y}dx}=\mu e^{-\mu y}(1-e^{-\lambda y}) ,y>0\\
&Z_1=5Y-2 \\
&f_{Z_2}(z)=\frac{1}{5}f_{Y_2}(\frac{y+2}{5}) \\
&f_{Z_2}(z)=\frac{\mu}{4}e^{-\frac{\mu (z+2)}{5}}(1-e^{-\frac{\lambda (z+2)}{5}}) , z \geq -2
\end{aligned}
\end{equation*}
综上,
\begin{equation*}
f_Z(z)=\begin{cases}
\frac{\mu}{4}e^{-\frac{\mu (z+2)}{5}}(1-e^{-\frac{\lambda (z+2)}{5}}) ,-2\leq z\leq 1 \\
\frac{\mu}{4}e^{-\frac{\mu (z+2)}{5}}(1-e^{-\frac{\lambda (z+2)}{5}})+\frac{\lambda}{4}e^{-\frac{\lambda (z-1)}{4}}(1-e^{-\frac{\mu (z-1)}{4}}) , z \geq 1
\end{cases}
\end{equation*}
\begin{equation*}
E(Z)=\frac{5}{\mu}+\frac{4}{\lambda}-\frac{3\mu}{5(\lambda + \mu)}+\frac{4\lambda -\mu}{(\lambda + \mu)^2}
\end{equation*}
\paragraph{}
4.假设股票市场每一天都是交易日,每个交易日的收盘股指点数是独立同分布的随机变量。从某个起始点开始算起,称一个交易日为拐点,如果从起始日到该交易日间每天收盘股指点数都在上涨,紧随该交易日后的后一天收盘股指点数下跌。请计算拐点到起始日之间的天数的均值。
设第i天的收盘股指点数为$X_i,X_i$的累积分布函数为$F(x)$,起始日与拐点之间的天数为Y。此处定义:如果第1天到第5天一直上涨,第6天下降,那么$Y=4$。
(1)排列思想
设事件A为“收盘股指点数从起始点到第$k+1$天一直上升”,事件B为“收盘股指点数从起始点到第$k+2$天一直上升”。则“第k天是拐点”等价于$A-B$。
\begin{equation*}
\begin{aligned}
P(A)&=\frac{1}{(k+1)!}\\
P(B)&=\frac{1}{(k+2)!}\\
P(Y=k)&=P(A)-P(B)\\
&=\frac{1}{(k+1)!}-\frac{1}{(k+2)!}\\
E(Y)&=\sum_{k=1}^\infty{\frac{(k+1)(k+2)}{(k+2)!}-2\times \frac{k+1}{(k+2)!}}\\
&=\sum_{k=1}^{\infty}{\frac{1}{k!}-2\times \frac{k+2-1}{(k+2)!}}\\
&=\sum_{k=1}^{\infty}{\frac{1}{k!}-2\times \frac{1}{(k+1)!} + 2\times \frac{1}{(k+2)!}}\\
&=e-(2e-2)+2(e-1-\frac{1}{2})\\
&=e-1
\end{aligned}
\end{equation*}
(2)分布函数法
\begin{equation*}
\begin{aligned}
P(Y=k)&=\int_0^\infty f(x_{k+1})dx_{k+1}\int_0^{x_{k+1}} f(x_{k+2})dx_{k+2}\int_0^{x_{k+1}} f(x_k)dx_k\int_0^{x_k} f(x_{k-1})dx_{k-1}\cdots \int_0^{x_2} f(x_1)dx_1\\
&=\int_0^\infty f(x_{k+1})dx_{k+1}\int_0^{x_{k+1}} f(x_{k+2})dx_{k+2}\times \frac{1}{k!}[F(x_{k+1})]^k\\
&=\int_0^\infty \frac{1}{k!}[F(x_{k+1})]^{k+1}f(x_{k+1})dx_{k+1}\\
&=\frac{1}{k!(k+2)}
\end{aligned}
\end{equation*}
求均值同上。
\paragraph{}
5.设X,Y均为连续随机变量。X有密度函数$f_X(x)={\lambda}^2xe^{-\lambda x}$,而Y满足[0,X]上的均匀分布,计算$E(Y^2\lvert Y>1)$。
\begin{equation*}
\begin{aligned}
f_{Y\lvert X}(y\lvert x)&=\frac{1}{x},0<y<x\\
f_{XY}(x,y)&=f_X(x)f_{Y\lvert X}(y\lvert x)\\
&={\lambda}^2e^{-\lambda x},0<y<x\\
f_{Y}(y)&=\int_y^\infty {\lambda}^2e^{-\lambda x}dx\\
&=\lambda e^{-\lambda y},y>0\\
f_{Y\lvert Y>1}(y\lvert y>1)&=\frac{\lambda e^{-\lambda y}}{\int_1^\infty {\lambda e^{-\lambda y}dy}}\\
&=\lambda e^{\lambda -\lambda y},y>1\\
E(Y^2\lvert Y>1)&=\int_1^\infty {\lambda y^2 e^{\lambda - \lambda y}dy}\\
&=1+\frac{2}{\lambda}+\frac{2}{\lambda ^2}
\end{aligned}
\end{equation*}
\paragraph{}
6.设X,Y服从均值为1,方差为0的高斯分布,且相互独立。试求:$E[X^2+Y^2\lvert cos(\frac{X}{Y})]$。
由书后习题可知,$X^2+Y^2$与$\frac{X}{Y}$相互独立,所以$X^2+Y^2$与$cos(\frac{X}{Y})$相互独立。
\begin{equation*}
E[X^2+Y^2\lvert cos(\frac{X}{Y})]=E(X^2 +Y^2)=E(X^2)+E(Y^2)=Var(X)+E^2(X)+Var(Y)+E^2(Y)=2
\end{equation*}
\paragraph{}
7.连续投掷一枚不均匀的硬币N次,正面向上的概率为p,其中N服从参数为$\lambda$的Poisson分布。设正面向上的次数为M,反面向上的次数为K,计算$E[min{M,K}]$。
答案链接:https://math.stackexchange.com/questions/2198813/properties-of-the-minimum-of-two-poisson-random-variables
麻烦看懂了的同学教教我,谢谢!
\paragraph{}
8.考虑一枚不均匀的硬币,已知存在随机变量N,使得连续抛掷N次,第N次的结果满足出现正面和反面的概率均为$\frac{1}{2}$。因此,即使是不公平硬币,也可以用于公平游戏。请给出N的具体描述,并通过计算验证其公平性。

每组连续丢两次硬币,直到第k组出现“正反”或“反正”的组合,此时丢的两次为第2k-1次和第2k次。那么N=2k。
公平性:$P(+\lvert N) = P(-\lvert N)$。即第N次为+的组合模式(……-+)与第N次为-的组合模式(……-+)一一对应,对应的事件概率相同。
\paragraph{}
9.设随机变量X和Y满足$E(X)=E(Y)=1$,且有$E(X\lvert Y)=E(X)$,计算$E(XY)$。很明显,如果X和Y相互独立,一定有$E(X\lvert Y)=E(X)$。反之成立吗?请举例说明。
\begin{equation*}
E(XY)=E(E(Xy\lvert Y=y))=E(YE(X\lvert Y))=E(YE(X))=E(Y)=1
\end{equation*}
举例说明:
\begin{tabular}{|l|c|c|c|}
\hline
\diagbox{Y}{X} & 0 & 1 & 2 \\
\hline
0 & $\frac{1}{4}$ & $\frac{1}{8}$ & $\frac{1}{8}$ \\
\hline
1 & $\frac{1}{8}$ & $\frac{3}{8}$ & 0 \\
\hline
\end{tabular}
\paragraph{}
10.篮球赛时长n分钟,某球员每一分钟有一次投篮机会,投中概率为p,并且教练规定,若一次投篮不中,下一分钟不得投篮,需将机会交给队友。请计算一场球赛中该队员的投中次数的均值。
\begin{equation*}
\begin{aligned}
E_1&=p\\
E_2&=p^2+p\\
E_n&=p(1+E_{n-1})+(1-p)E_{n-2}\\
D_n&=E_n-E_{n-1}+\frac{p}{p-2}\\
D_n&=(p-1)D_{n-1}\\
D_n&=(p^2+\frac{p}{p-2})(p-1)^{n-2}, n\geq 2\\
E_n&=E_1+\sum_{k=2}^n{D_n}+(n-1)\times \frac{p}{p-2}
\end{aligned}
\end{equation*}
\begin{equation*}
E_n=\begin{cases}
p, n=1\\
p^2+(p^2+\frac{p}{p-2})\frac{1-(p-1)^{n-1}}{2-p}-(n-1)\frac{p}{p-2}, n \geq 2
\end{cases}
\end{equation*}
参考文献:https://cowandsheep.github.io/Fishtoucher/$\#$/
\end{document}


