\documentclass[UTF8]{ctexart}
\usepackage{subfigure}
\usepackage{caption}
\usepackage{amsmath}
\usepackage{amssymb}
\usepackage{pifont}
\usepackage{geometry}
\usepackage{graphicx}
\usepackage{gensymb}
\usepackage{wrapfig}
\usepackage{titlesec}
\usepackage{float}
\usepackage{diagbox}
\usepackage{fancyhdr}
\pagestyle{plain}
\geometry{a4paper,scale=0.8}
\CTEXsetup[format+={\raggedright}]{section} 
\title{随机过程2020-2021期末}
\author{Deschain}
\titlespacing*{\section}
{0pt}{0pt}{0pt}
\titlespacing*{\subsection}
{0pt}{0pt}{0pt}
\titlespacing*{\paragraph}
{0pt}{0pt}{0pt}
\titlespacing*{\subparagraph}
{0pt}{0pt}{0pt}
\titleformat*{\section}{\normalsize}
\begin{document}
\maketitle
\section*{1.}
\begin{equation*}
  \begin{bmatrix}
    0           & 1           & 0           & 0 & 0 \\
    0           & 0           & 1           & 0 & 0 \\
    1           & 0           & 0           & 0 & 0 \\
    \frac{1}{3} & \frac{1}{3} & \frac{1}{3} & 0 & 0 \\
    \frac{1}{3} & \frac{1}{3} & \frac{1}{3} & 0 & 0 \\
  \end{bmatrix}
\end{equation*}
\section*{2.}
\begin{equation*}
  P=\begin{bmatrix}
    \frac{3}{10} & \frac{1}{2}   & \frac{1}{5}   \\
    \frac{3}{7}  & \frac{19}{70} & \frac{3}{10}  \\
    \frac{3}{20} & \frac{21}{80} & \frac{47}{80} \\
  \end{bmatrix},\quad
  \pi=\pi\cdot P,\quad\therefore\pi=[\frac{2}{7},\frac{1}{3},\frac{8}{21}]
\end{equation*}
\section*{3.}
 (a)
\begin{equation*}
  \begin{aligned}
     & Z_n=Y(n)=Z(n-1)+Y(1),\quad\therefore Z_n\sim Markov                                 \\
     & P(\sum\limits_{k=0}^N X_k=m)=(\frac{1}{2})^N\tbinom{N}{\frac{m+N}{2}}               \\
     & P(N(t)=N)=\frac{\lambda^N}{N!}e^{-\lambda}                                          \\
     & P_{ij}(1)=\sum\limits_{m=\lvert j-i\rvert}^{\infty}\frac{\lambda^m}{m!}e^{-\lambda}
    (\frac{1}{2})^m\tbinom{m}{\frac{m+j-i}{2}}                                             \\
  \end{aligned}
\end{equation*}
$\{Z_n\}$各状态相通,常返性相同,以下计算状态0的常返性。
\begin{equation*}
  \begin{aligned}
     & P_{00}(n)=\sum\limits_{n=0}^\infty\sum\limits_{j=0}^\infty\frac{(n\lambda)^{2j}}{2j!}
    e^{-\lambda n}\tbinom{2j}{j}(\frac{1}{2})^{2j}
    =\sum\limits_{j=0}^\infty\frac{\tbinom{2j}{j}}{(2j)!}\lambda^{2j}\sum\limits_{n=0}^\infty
    n^{2j}e^{-\lambda n}
    \approx\sum\limits_{j=0}^\infty\frac{\tbinom{2j}{j}}{(2j)!}\int_0^\infty x^{2j}e^{-\lambda x}dx \\
     & =\frac{1}{\lambda}\sum\limits_{j=0}^\infty\tbinom{2j}{j}(\frac{1}{4})^{j}
    \approx\frac{1}{\lambda}\sum\limits_{j=0}^\infty(\frac{1}{4})^{j}
    \frac{(\frac{2j}{e})^{2j}\sqrt{4\pi j}}{(\frac{j}{e})^{2j}2\pi j}
    =\frac{1}{\lambda}\sum\limits_{j=0}^\infty(\pi j)^{-\frac{1}{2}}\to\infty
  \end{aligned}
\end{equation*}
$\therefore$是常返的。
\section*{4.}
\begin{equation*}
  \begin{aligned}
    (a) & P(T>t)=e^{-\lambda_At}+(1-e^{-\lambda_A t})e^{\lambda_B(t-1)}=e^{-\lambda_A t}
    +e^{-\lambda_B(t-1)}-e^{-(\lambda_A+\lambda_B)t+\lambda_B},\quad t>1                   \\
        & F_T(t)=1+e^{-(\lambda_A+\lambda_B)t+\lambda_B}-e^{-\lambda_At}
    -e^{-\lambda_B(t-1)},\quad t>1                                                         \\
        & f_T(t)=-(\lambda_A+\lambda_B)e^{-(\lambda_A+\lambda_B)t+\lambda_B}
    \lambda_Ae^{-\lambda_At}+\lambda_Be^{-\lambda_B(t-1)},\quad t>1                        \\
        & E(T)=\int_0^\infty tf_T(t)dt=e^{-\lambda_A}(\frac{1}{\lambda_A}
    -\frac{1}{\lambda_A+\lambda_B})+1+\frac{1}{\lambda_B}                                  \\
    (b) & P(A+B=10)=\sum\limits_{k=0}^{10}\frac{(2\lambda_A)^k\lambda_B^{10-k}}{k!(10-k)!}
    e^{-2\lambda_A-\lambda_B}                                                              \\
        & P(A=4,B=6)=\frac{(2\lambda_A)^4\lambda_B^6}{4!\times6!}e^{-2\lambda_A-\lambda_B} \\
        & P(A=4\lvert A+B=10)=\frac{\frac{(2\lambda_A)^4\lambda_B^6}{4!\times6!}}
    {\sum\limits_{k=0}^{10}\frac{(2\lambda_A)^k\lambda_B^{10-k}}{k!(10-k)!}}               \\
  \end{aligned}
\end{equation*}
\section*{5.}
 (a)设每秒逃逸的粒子数为$N$,逃逸速率$P_1=(1-p)^k$
\begin{equation*}
  \therefore N(t)\sim Poisson(p_1\lambda),\quad \lambda=100,\quad E[N]=100(1-p)^k
\end{equation*}
(b)设总辐射强度为$Y(t)$,记$N(t)\sim Poisson(\lambda_1).\lambda_1=100(1-p)^kP_n(k)$,其中
$P_n(k)$是$n$个防护罩中有$k$个打开的概率。
\begin{equation*}
  \begin{aligned}
     & Y(t)=\sum\limits_{k=0}^{N(t)}V_k(t,t_0),\quad E[V_1(t,t_0)]=3e^{\mu(t-t_0)},\quad t\geq t_0       \\
     & E[Y(t)]=\lambda_1\int_0^t3e^{\mu(t-t_0)}dt_0=\frac{3\lambda_1}{\mu}(e^{\mu t}-1)
    =\frac{3\lambda_1}{\mu}(e^{10\mu}-1),\quad E[V_1^2(t,t_0)]=\frac{4}{3}e^{\mu(t-t_0)},\quad t\geq t_0 \\
     & Var[Y(t)]=\lambda_1\int_0^t E[V_1^2(t,t_0)]dt_0
    =\lambda_1\int_0^t\frac{4}{3}e^{2\mu(t-t_0)}dt_0=\frac{2\lambda_1}{3\mu}(e^{20\mu}-1)                \\
  \end{aligned}
\end{equation*}
\end{document}